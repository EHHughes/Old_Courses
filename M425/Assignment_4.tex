\documentclass{article}

\title{	
	\normalfont\normalsize 
	\rule{\linewidth}{0.5pt}\\ % Thin top horizontal rule
	\vspace{14pt} % Whitespace
	{\LARGE MATH425 Assignment 4\\ % The assignment title
    \large \textit{} \\}
	\vspace{6pt} % Whitespace
	\rule{\linewidth}{1pt}\\ % Thick bottom horizontal rule
}

\author{Elliott Hughes}
\date{\normalsize\today}
\usepackage{tikz}
\usetikzlibrary{arrows,automata}
\usetikzlibrary{positioning}
\usetikzlibrary{arrows.meta,positioning}
\usepackage{mdframed}
\usepackage{amsmath}
\usepackage{amssymb}
\usepackage{graphicx}
\graphicspath{ {./Images/} }
\usepackage{commath}
\usepackage{textcomp}
\usepackage{gensymb}
\usepackage{float}
\usepackage{hyperref}
\usepackage[margin=1in]{geometry}
\usepackage{caption}
\usepackage{subcaption}
\usepackage{sectsty}
\usepackage{titlesec}
\usepackage{mathrsfs}

\begin{document}

\maketitle

\section*{W6Q3}
Because $f$ is continuous and $f(0) = 0$, we can choose a $\delta > 0$ such that $\forall x \in B(0,\delta)$ 
we have $|f(x)| < \epsilon$, $\epsilon < 1$. Then since all polynomials are continuous, for 
every $k \in \mathbb{N}$ we have that $(f(z))^k$ is continuous and $|f(z)|^k < \epsilon^k$. 
Therefore for all $\{a_k\}_{k=0}^\infty$ bounded by $M > 0$ we have

\begin{equation*}
    \left|\sum_{k=0}^\infty a_k(f(z))^k\right| \leq \sum^\infty_{k=0}|a_k||f(z)|^k \leq \sum_{k=0}^\infty M\epsilon^k = M\frac{\epsilon}{1-\epsilon}
\end{equation*}

Thus every function in this family is uniformly bounded and converges uniformly by direct application 
of the Weierstrass M-test. Since convergence is uniform, it follows that each function in the family 
is continuous by the properties of uniform convergence (each term in the series for that function will be continuous, so each 
partial sum will be continuous and thus the limiting function will also be continuous). Thus this family of continuous functions is 
uniformly bounded.


\section*{W7Q3}
It will be convenient to show that $h(x) = \sqrt{x + 1}$ is a contraction mapping. Assuming $\mathbb{R}$ 
is equipped with the usual metric we have that (for $x,y \geq 0$) 

\begin{align*}
	|\sqrt{x+1} - \sqrt{y+1}| &= \frac{\sqrt{x+1} + \sqrt{y+1}}{\sqrt{x+1} + \sqrt{y+1}}|\sqrt{x+1} - \sqrt{y+1}| \\
	& = \left|\frac{x+1 -y - 1}{\sqrt{x+1} + \sqrt{y+1}}\right| \\
	&= \frac{|x-y|}{\sqrt{x+1} + \sqrt{y+1}} \\
	& \leq \frac{1}{2}|x-y|
\end{align*}

So $h$ is a contraction mapping. Furthermore, it is useful to note that $h(x) \geq 0$ for all $x \geq 0$. 
Let $h^n$ denote repeated composition of the function $h$ so $h^2 = h \circ h$, etc (it is also 
convenient to let $h^0(x) = x$). These functions constructed by repeated composition are from $\mathbb{R}$ to $\mathbb{R}$ if $x \geq 0$ as $h(x) \geq 0$ 
for all $x \geq 0$. Then 
we can rewrite $|f_n(x) - f_n(y)| = |h^{n-1}(f_1(x)) - h^{n-1}(f_1(y))|$ for $x,y \in [0,1]$, $n \in \mathbb{N}$. 
Then $f_1(x) = x$ so $|f_n(x) - f_n(y)| = |h^{n-1}(x) - h^{n-1}(y)| \leq |x-y|$ as $h$ is a 
contraction mapping. Therefore, $\forall \epsilon > 0$ take $\delta = \epsilon$. If $x,y \in [0,1]$ and 
$|x-y| < \delta = \epsilon$, $|f_n(x) - f_n(y)| \leq |x-y| < \delta = \epsilon$ for all $n \in \mathbb{N}$ 
and so this family of functions is equicontinuous.

\end{document}