\documentclass{article}

\title{	
	\normalfont\normalsize 
	\rule{\linewidth}{0.5pt}\\ % Thin top horizontal rule
	\vspace{14pt} % Whitespace
	{\LARGE MATH425 Assignment 6\\ % The assignment title
    \large \textit{} \\}
	\vspace{6pt} % Whitespace
	\rule{\linewidth}{1pt}\\ % Thick bottom horizontal rule
}

\author{Elliott Hughes}
\date{\normalsize\today}
\usepackage{tikz}
\usetikzlibrary{arrows,automata}
\usetikzlibrary{positioning}
\usetikzlibrary{arrows.meta,positioning}
\usepackage{mdframed}
\usepackage{amsmath}
\usepackage{amssymb}
\usepackage{graphicx}
\graphicspath{ {./Images/} }
\usepackage{commath}
\usepackage{textcomp}
\usepackage{gensymb}
\usepackage{float}
\usepackage{hyperref}
\usepackage[margin=1in]{geometry}
\usepackage{caption}
\usepackage{subcaption}
\usepackage{sectsty}
\usepackage{titlesec}
\usepackage{mathrsfs}

\begin{document}

\maketitle

\section*{W10Q1}
It will be useful to show that $f_n(x)$ is bounded for $n \geq 2$. Clearly $f_{n+1}$ is well-defined if 
$f_n$ is well-defined and has no zeros on $(0,\infty)$. Since $f_1(x) = x$ 
is well-defined on the whole interval each $f_n$ is also well-defined by induction. It is 
hopefully obvious that if $f_n(x) > 0$ on the interval then $f_{n+1}(x) > 1 > 0$ on the interval 
and since $f_1(x) > 0$ for $x \in (0,\infty)$ it follows by induction that $f_n(x) > 1$ on 
the interval for all $n \geq 1$. For $n \geq 2$ we have $f_n(x) = 1 + 1/f_{n-1}(x)$ and 

\begin{equation*}
    f_{n+1}(x) = 1 + \frac{1}{f_n(x)} = 1 + \frac{1}{1 + \frac{1}{f_{n-1}(x)}} = 1 + \frac{f_{n-1}(x)}{f_{n-1}(x) + 1} \leq 2
\end{equation*}

For $n > 2$ let 

\begin{equation*}
    d = \sup_{x \in (0,\infty)}\left\{\left|f_n(x) - \frac{1+\sqrt{5}}{2} \right|\right\}
\end{equation*}

Then for any $x \in (0,\infty)$ we have 

\begin{align*}
    \left|f_{n+1}(x) - \frac{1+\sqrt{5}}{2}\right| &= \left| 1 + \frac{1}{f_n(x)} - \frac{1+\sqrt{5}}{2} \right| \\
    &= \left| \frac{1}{f_n(x)} + \frac{1-\sqrt{5}}{2} \right| \\
    &= \left|\frac{2 + f_n(x)(1-\sqrt{5})}{2f_n(x)}\right| \\
    &\leq \frac{1}{2}|2 + f_n(x)(1-\sqrt{5})| \\
    &= \frac{1}{2}\left|2 + \left(f_n(x) - \frac{1 + \sqrt{5}}{2}+ \frac{1 + \sqrt{5}}{2}\right)(1-\sqrt{5}) \right|\\
    &= \frac{1}{2}\left|2  + \frac{1}{2}(1+\sqrt{5})(1-\sqrt{5}) + \left(f_n(x) - \frac{1 + \sqrt{5}}{2}\right)(1-\sqrt{5})\right|\\
    &=\frac{1}{2}\left|\left(f_n(x) - \frac{1 + \sqrt{5}}{2}\right)(1-\sqrt{5}) \right|
\end{align*}

It follows from this that 

\begin{align*}
    \sup_{x \in (0,\infty)}\left\{\left|f_{n+1}(x) - \frac{1+\sqrt{5}}{2} \right|\right\} \leq \sup_{x \in (0,\infty)}\left\{\left|\frac{1-\sqrt{5}}{2}\left(f_n(x) - \frac{1 + \sqrt{5}}{2}\right)\right|\right\}
\end{align*}

Consider the sequence

\begin{equation*}
    \{M_n\}_{n=1}^{\infty}, \qquad M_n = \sup_{x \in (0,\infty)}\left\{\left|f_{n+2}(x) - \frac{1+\sqrt{5}}{2} \right|\right\}
\end{equation*}

Since $f_3$ is bounded, $M_1$ exists and then for all $n \geq 1$, $M_n \leq \left(\frac{|1 - \sqrt{5}|}{2}\right)^{n-1}M_1$. 
Then $\frac{|1 - \sqrt{5}|}{2} < 1$ and $M_n \geq 0$ for all $n$ by construction, so the squeezing 
theorem implies that $M_n$ must converge to zero. Therefore for all $\epsilon > 0$ one can choose 
$N \in \mathbb{N}$ such that $M_n < \epsilon$ which implies that for all $x \in (0,\infty)$ and 
$n > N+2$ we have

\begin{equation*}
    \left|f_n(x) - \frac{1 + \sqrt{5}}{2}\right| < \epsilon
\end{equation*}

Thus $f_n(x)$ converges uniformly. While the proof that this sequence converges uniformly is 
somewhat involved, it is straightforward to demonstrate that this is not equicontinuous. In 
particular, it is sufficient to demonstrate that one of the functions in this 
family is not uniformly continuous. Consider $f_2(x) = 1 + 1/x$. Then for $x,y \in (0,\infty)$ 
we have 

\begin{equation*}
    |f_2(x) - f_2(y)| = \left|1 + \frac{1}{x} - 1 - \frac{1}{y} \right| = \frac{|x - y|}{xy}
\end{equation*}

Let $\epsilon = 1/2$. Then for any $1 > \delta > 0$ we can choose $x,y \in (0,\infty)$ such that $\delta/2 <|x -y| < \delta$ 
and $x,y < \delta$. This leads to 

\begin{equation*}
    |f_2(x) - f_2(y)| = \frac{|x - y|}{xy} > \frac{\delta}{2\delta^2} = \frac{1}{2\delta} > \frac{1}{2}
\end{equation*}

So this function is not uniformly continuous and the family as a whole is not equicontinuous.

\subsection*{W11Q2}
Note that $|\sin(z) - 3z^2 -1 + 3z^2 +1| \leq |\sin(z)-3z^2-1| + |3z^2+1|$. Furthermore, this 
inequality will be strict at every point except where $|\sin(z) - 3z^2-1| = |\sin(z)| - |3z^2+1|$. 
If $z_0$ is such a point then a trivial rearrangement of the above condition leads to 

\begin{equation*}
    \left|\frac{\sin(z_0)}{3z_0^2+1} -1 \right| = \left|\frac{\sin(z_0)}{3z_0^2+1}\right| - 1
\end{equation*}

Let the real and imaginary parts of this fraction be given by $\alpha$ and $\beta$ respectively. 
The above condition can then be manipulated as follows

\begin{align*}
    \sqrt{(\alpha-1)^2 + \beta^2} &= \sqrt{\alpha^2 + \beta^2} - 1\\
    \alpha^2 + 1 + \beta^2 - 2\alpha &= \alpha^2 + \beta^2 +1 -2\sqrt{\alpha^2+\beta^2} \\
    1-2\alpha &= 1 - 2\sqrt{\alpha^2 + \beta^2} \\ 
    \alpha^2 &= \alpha^2 + \beta^2 \\
    \beta^2 &= 0
\end{align*}

So in order for this to occur the imaginary part of the fraction must be zero. It is hopefully obvious 
that an additional necessary condition is that the fraction must be greater than one. However 
on the unit disc the ratio of $\sin(z_0)$ and $3z_0^2+1$ is bounded above by 

\begin{align*}
    \sup_{z_0 \in \partial B(0,1)}\left\{\left|\frac{\sin(z_0)}{3z_0^2+1} \right|\right\} &\leq \sup_{z_0 \in \partial B(0,1)}\left\{\frac{|\sin(z_0)|}{2}\right\} \\
    &= \sup_{z_0 \in \partial B(0,1)}\left\{\frac{|\sinh(iz_0)|}{2}\right\} \\
    &=\sup_{z_0 \in \partial B(0,1)}\left\{\frac{|e^{iz_0}-e^{-iz_0}|}{4}\right\} \\
    &=\sup_{\theta \in [0,2\pi)}\left\{\frac{|e^{\sin(\theta)}+e^{-\sin(\theta)}|}{4}\right\} \\
    &\leq \frac{e+e^{-1}}{4} < 1
\end{align*}

So the ratio must be less than 1 and the inequality is strict on the boundary of the unit disc. Therefore Rouche's theorem leads to 
the following equality

\begin{equation*}
    \frac{1}{2\pi i}\int_{\partial B(0,1)}\frac{\cos(\zeta) -6\zeta}{\sin(\zeta) - 3\zeta^2-1}d\zeta = \frac{1}{2\pi i}\int_{\partial B(0,1)}\frac{6\zeta}{3\zeta^2+1}
\end{equation*}

Clearly $3z^2+1$ is non-vanishing on the boundary. Furthermore, a previous calculation implies that 

\begin{align*}
    \sup_{z_0 \in \partial B(0,1)}\{|\sin(z_0)\} < 2 &= \inf_{z_0 \in \partial B(0,1)}\{3z_0^2+1\} \\
    \implies \sup_{z_0 \in \partial B(0,1)}\{|3z_0^2+1-\sin(z_0)|\} &> 0 \\
    \implies \sup_{z_0 \in \partial B(0,1)}\{|\sin(z_0)-3z_0^2-1|\} &> 0 \\
\end{align*}

So both of these are non-vanishing on the boundary. Thus we obtain that the number of zeros (counting multiplicities) 
on the unit disc is the same as the number of roots of $3z^2+1$. Since there are clearly two 
roots on the unit disc it follows that the number of zeros of $\sin(z) -3z^2-1$ (counting multiplicities) 
is two.

\paragraph{}
Finally to find the number of unique solutions, we must ensure that this is not a double root of 
the function $\sin(z)-3z^2-1$. If a zero of this function were a double root, this would imply that 
the derivative of this function $\cos(z)-6z$ would also be zero at this point. It is straightforward to 
repeat the analysis conducted above and see that 

\begin{equation*}
    \sup_{z_0 \in \partial B(0,1)}\left\{\left|\frac{\cos(z_0)}{6z}\right|\right\} \leq \frac{e+e^{-1}}{12}
\end{equation*}

So (by an identical argument to that above) both $\cos(z)-6z$ and $6z$ are non-vanishing on the boundary of the unit disk and we can use Rouche's theorem to 
show that the number of zeros of $\cos(z)-6z$ and $6z$ (counting multiplicities) must be the same 
on the interior of the disk. Since $6z$ has only one zero on the interior of the unit disk it follows 
that the derivative of $\sin(z)-3z^2-1$ must also have one zero in this region. Since $\cos(x)$ and 
$6x$ intersect on the interval $[0,1)$ the only zero of the derivative must occur on the real line 
between zero and one. However, on the real line $\sin(z) < 3z^2+1$ everywhere so there cannot be a 
zero of the function in this segment. 

\paragraph{}
Therefore the zeros of $\sin(z)-3z^2-1$ must be of order 1 and thus distinct. It follows from this 
that there are two zeros of this function and thus two solutions of the equation $\sin(z) = 3z^2+1$ 
on the interior of the unit disk.


\end{document}