\documentclass{article}

\title{	
	\normalfont\normalsize 
	\rule{\linewidth}{0.5pt}\\ % Thin top horizontal rule
	\vspace{14pt} % Whitespace
	{\LARGE MATH425 Assignment 1\\ % The assignment title
    \large \textit{} \\}
	\vspace{6pt} % Whitespace
	\rule{\linewidth}{1pt}\\ % Thick bottom horizontal rule
}

\author{Elliott Hughes}
\date{\normalsize\today}
\usepackage{tikz}
\usetikzlibrary{arrows,automata}
\usetikzlibrary{positioning}
\usetikzlibrary{arrows.meta,positioning}
\usepackage{mdframed}
\usepackage{amsmath}
\usepackage{amssymb}
\usepackage{graphicx}
\graphicspath{ {./Images/} }
\usepackage{commath}
\usepackage{textcomp}
\usepackage{gensymb}
\usepackage{float}
\usepackage{hyperref}
\usepackage[margin=1in]{geometry}
\usepackage{caption}
\usepackage{subcaption}
\usepackage{sectsty}
\usepackage{titlesec}

\begin{document}

\maketitle

\section*{Q4}
\subsection*{(a)}
Note at $x=0$, $sin(n(0)) = 0$. At $x\neq 0$ we have 

\begin{equation}
	\left|\frac{sin(nx)}{1+n^2x^2}\right| < \left|\frac{1}{1+n^2x^2}\right| < \left(\frac{1}{nx}\right)^2
\end{equation}

Then $\forall \epsilon > 0 \quad\exists N \in \mathbb{N}$ s.t. $\forall n \geq N \implies \frac{1}{n^2} < x^2\epsilon$.
So $\forall x \in \mathbb{R}$, $\forall \epsilon > 0 \quad\exists N \in \mathbb{N}$ s.t. $\forall n \leq N$

\begin{equation*}
	\left|\frac{sin(nx)}{1+n^2x^2} - 0 \right| < \epsilon
\end{equation*}

So $f_n(x)$ converges pointwise to zero.

\subsection*{(b)}
Consider the sequence of points $\{1/n\}_{n=1}^\infty$. At each point in this sequence we have that 

\begin{equation*}
	\frac{sin(1)}{1 + 1/n} \neq 0
\end{equation*}

And in the limiting case $f_n(1/n) \rightarrow sin(1) \neq 0$. So this sequence does not converge 
uniformly as $\exists x_* \in \mathbb{R}$ s.t. $\exists \epsilon < sin(1)/2$ s.t. $\forall n \in \mathbb{N}$, $n >0$ 
we have that $|f_n(x_*)| > \epsilon$.

\section*{Q5}
\subsection*{(a)}
Note for $r \in (0,1) \quad \forall \epsilon > 0 \quad \exists N \in \mathbb{N}$ s.t. $\forall n \geq N$ we 
have $r^{2n} < \epsilon$. Thus $\forall r \in (0,1) \quad \exists N \in \mathbb{N}$ s.t. $\forall n \geq N \implies 
r^{2n} < 1/2 \implies 1 < 2(1-r^{2n}) \implies 1/(1-r^{2n}) < 2$ as required.

\subsection*{(b)}
Note that at $x = 1,-1$ each term in the series is undefined so the infinite sum is thus also 
undefined. Considering the separate subdomain in turn, we will first turn to $x \in (-1,1)$. 
Inside this domain, we can apply the ratio test to determine the series' convergence.

\begin{equation*}
	R_n(x) = \left|\left(\frac{x^{n+1}}{1-x^{2n+2}}\right)/\left(\frac{x^n}{1-x^{2n}}\right)\right| = \left|\frac{x(1-x^{2n})}{1-x^{2n+2}}\right| = \left|\frac{x-x^{2n+2}}{1-x^{2n+2}}\right|
\end{equation*}

Since $x \in (-1,1)$ in the limit $R_n(x)$ will clearly approach $|x| < 1$ and so by the ratio test we 
have that $R_n(x)$ converges pointwise when $x \in (0,1)$. For $x > 1$ we have 

\begin{equation*}
	\left|\frac{x^n}{1-x^{2n}}\right| = \left|\frac{1}{(1-x^n)(1+x^n)}\right| < \left|\frac{1+x^n}{(1-x^n)(1+x^n)}\right| = \left|\frac{1}{1-x^n}\right|
\end{equation*}

Then we can apply the ratio test to see that 

\begin{equation*}
	R_n(x) = \left|\left(\frac{1}{1 - x^{n+1}}\right)/\left(\frac{1}{1 - x^{n}}\right)\right| = \left|\frac{1 - x^n}{1 - x^{n+1}}\right|
\end{equation*}

Since $x > 1$ it follows that in the limit that $R_n(x)$ for this series will converge to $1/x < 1$. 
Therefore the series $\sum_{n=1}^{\infty}|1/(1-x^n)|$ converges by the ratio test.
Thus the series $|S(x)| = \sum_{n=1}^{\infty}|x^n/(1-x^{2n})|$ is bounded above for all $x > 1$ and 
(since each individual term is strictly positive) it must converge to a finite limit by the 
completeness axiom. Since $\sum_{n=1}^{k}|x^n/(1-x^{2n})| \geq \left|\sum_{n=1}^{k}x^n/(1-x^{2n})\right|$ 
we obtain absolute convergence for any $x$ s.t. $|x| > 1$. Therefore the set $E$ on which 
$S(x)$ converges pointwise is $E = \{x \in \mathbb{R}:|x| \neq 1\}$.

\paragraph{}
We must now consider the question of uniform convergence. 
Note that each individual term has the derivative 

\begin{align*}
	f'_n(x) &= \frac{nx^{n-1}(1-x^{2n}) + 2nx^{2n-1}x^n}{(1-x^{2n})^2} \\
	&= \frac{nx^{n-1}-nx^{3n-1} + 2nx^{3n-1}}{(1-x^{2n})^2} \\
	&= \frac{nx^{n-1}+nx^{3n-1}}{(1-x^{2n})^2} \\
	&= \frac{nx^{n-1}(1 + x^{2n})}{(1-x^{2n})^2}
\end{align*}

So for a closed interval $[a,b]$ inside the domain of the series and with $a > 0$
 the absolute value of each term will be maximized either at 
one of the end points or at $x=0$ (as this is the only point where $f'_n = 0$) if this is inside the closed interval. Since the series 
converges absolutely pointwise we can thus set $M_n$ equal to the sum of the absolute value of each term at both endpoints and at $x=0$ (if this 
is inside the interval) and obtain a series $\sum_{n=1}^{\infty}M_n$. This series must be convergent 
as the sum of two or more convergent series is itself convergent. Then it follows that for each 
point inside the closed interval we have $|x^n/(1-x^{2n})| < M_n$ and so the Weierstrass M-test 
can be applied directly to show that the series converges uniformly inside this interval. Then 
since convergence for every interval $[a,b]$ in the domain of the series with $a > 0$ implies absolute 
convergence for any interval in the domain of the series (see the argument for pointwise 
convergence above) the series converges uniformly in every 
closed interval inside the domain of $S(x)$.

\subsection*{(c)}
Note from above that we have (for $x > 1$)

\begin{equation*}
	\left|\frac{x^n}{1-x^{2n}}\right| < \left|\frac{1}{1-x^n}\right| < \left|\frac{1}{(1-x)^n}\right| = \left|\frac{1}{1-x}\right|^n = \left(\frac{1}{x-1}\right)^n
\end{equation*}

So this series converges to $\delta/(1-\delta)$ where $\delta = |1/(1-x)|$ by the properties of 
geometric series. Note that the function $g(x) = 1/(x-1)$ has a range that spans all the 
positive reals and that it is clearly monotonically decreasing.
Then for all $\epsilon > 0$ choose $x_* > 1$ s.t. $\forall x \geq x_*$ we have that 
$1/(x-1) \leq \delta_*$ where

\begin{equation*}
	\delta_* < \frac{\epsilon}{1+\epsilon}
\end{equation*}

Thus for every $\epsilon > 0 \quad \exists x_*$ s.t. $\forall x > x_*$ we have 

\begin{equation*}
	\left|\sum_{n=1}^{\infty}\frac{x^n}{1-x^{2n}}\right|<\sum_{n=1}^{\infty}\left|\frac{x^n}{1-x^{2n}}\right| < \sum_{n=1}^{\infty}\left(\frac{1}{x-1}\right)^n \leq \frac{\delta_*}{1-\delta_*}<\epsilon
\end{equation*}

Therefore $\lim_{x \rightarrow \infty}S(x) = 0$. 

\end{document}