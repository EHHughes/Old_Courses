\documentclass{article}

\title{	
	\normalfont\normalsize 
	\rule{\linewidth}{0.5pt}\\ % Thin top horizontal rule
	\vspace{14pt} % Whitespace
	{\LARGE MATH411 Assignment 2\\ % The assignment title
    \large \textit{} \\}
	\vspace{6pt} % Whitespace
	\rule{\linewidth}{1pt}\\ % Thick bottom horizontal rule
}

\author{Elliott Hughes}
\date{\normalsize\today}
\usepackage{tikz}
\usetikzlibrary{arrows,automata}
\usetikzlibrary{positioning}
\usetikzlibrary{arrows.meta,positioning}
\usepackage{mdframed}
\usepackage{amsmath}
\usepackage{amssymb}
\usepackage{graphicx}
\graphicspath{ {./Images/} }
\usepackage{commath}
\usepackage{textcomp}
\usepackage{gensymb}
\usepackage{float}
\usepackage{hyperref}
\usepackage[margin=1in]{geometry}
\usepackage{caption}
\usepackage{subcaption}
\usepackage{sectsty}
\usepackage{titlesec}
\def\acts{\rotatebox[origin=c]{-60}{$\circlearrowright$}}

\begin{document}

\maketitle

\subsection*{4.1}
\begin{enumerate}
    \item This polynomial is irreducible, so we can use the technique presented in the notes. Calculating the discriminant we see that $D(f) = -4(-3)^3 - 27(1) = 9^2 \in \mathbb{Q}^2$. Therefore $\text{Gal}(f) = A_3$ 
    and so the action of the automorphisms on the roots is cyclic.
    \item Again $f(x)$ is irreducible, so we calculate the discriminant and find that $D(f) = -4(3)^3 - 27(1) = -5(3^3) \notin \mathbb{Q}^2$ so $\text{Gal}(f) = S_3$. 
    So $f$ has two complex and one real root and the action of $\text{Gal}(f)$ can be decomposed 
    into a combination of complex conjugation and/or a cyclic permutation of the roots.
    \item Clearly $\mathbb{Q}(\sqrt{2})$ and $\mathbb{Q}(\sqrt{3})$ are subfields of $\mathbb{Q}(\sqrt{2},\sqrt{3})$ 
    the splitting field of $f$. This then implies the existence of $H_1$ and $H_2$ subgroups of 
    order two in $\text{Aut}(\mathbb{Q}(\sqrt{2},\sqrt{3})|\mathbb{Q})$. Since $[\mathbb{Q}(\sqrt{2},\sqrt{3}):\mathbb{Q}] = 4$ 
    from the definition (an element in this extension field can clearly be written as $\alpha_1 + \alpha_2\sqrt{2} + \alpha_3\sqrt{3} + \alpha_4\sqrt{6}$ 
    for $\alpha_i \in \mathbb{Q}$) proposition 3.5 implies that $\#\text{Aut}(\mathbb{Q}(\sqrt{2},\sqrt{3})|\mathbb{Q}) \leq 4$. 
    The existence of two distinct subgroups of order 2 then clearly implies that $\text{Aut}(\mathbb{Q}(\sqrt{2},\sqrt{3})|\mathbb{Q}) \cong C_2 \times C_2$. 
    An element in the automorphism group can be written as a composition of an automorphism that sends 
    $\sqrt{2} \rightarrow -\sqrt{2}$ and an automorphism that sends $\sqrt{3} \rightarrow -\sqrt{3}$, 
    potentially with one or both of these automorphisms replaced with the identity automorphism.
    \item Suppose $E|\mathbb{F}_5$ contains $\alpha \in E$ such that $\alpha^2 =2$. Then 
    $(2\alpha)^2 = 4\alpha^2 \cong 3$ mod 5. Therefore an automorphism which permutes the roots of 
    $x^2 -2$ will also permute the roots of $x^2 -3$ in $E$. Thus $\text{Aut}(E|\mathbb{F}_5) \cong C_2$ 
    and the only non-identity automorphism in this group sends $\sqrt{2} \rightarrow - \sqrt{2}$ 
    and $\sqrt{3} \rightarrow -\sqrt{3}$.
    \item For $f = (x^2-2)(x^2-3) \in \mathbb{Q}(\sqrt{6})[x]$ the splitting field $\mathbb{Q}(\sqrt{2},\sqrt{3})$ is a
    quadratic extension of $\mathbb{Q}(\sqrt{6})$ (as $\sqrt{2}\sqrt{3} = \sqrt{6}$, so adding one of these 
    roots is sufficient to add the other). Consequently the automorphism group must be isomorphic 
    to $C_2$. In particular, the non-identity autmorphism must send $\sqrt{2} \rightarrow -\sqrt{2}$ 
    and since $\sqrt{6}$ is fixed it must also send $\sqrt{3} \rightarrow -\sqrt{3}$.
\end{enumerate}

\subsection*{4.2}
If $f(x) \in \mathbb{Q}[x]$ is irreducible and has degree $n$, then $\text{Gal}(f) = \text{Aut}(E|\mathbb{Q})$ (where $E|F$ is the splitting field of $f$) is a transitive 
subgroup of $S_n$ where $n = \text{deg}(f)$. Since $\mathbb{Q}$ is not characteristic 2 and the 
extension field contains complex elements, complex conjugation does not fix all elements in $E|\mathbb{Q}$ 
but does fix all elements in $\mathbb{Q}$. It is also obviously one-to-one and onto and preserves 
addition and multiplication, so complex conjugation is an automorphism of $E|\mathbb{Q}$.

\paragraph{}
Furthermore, since $\text{Aut}(E|\mathbb{Q})$ is a transitive subgroup of $S_n$ and $f$ has 
real and complex roots, for $\alpha$ a real root and $\beta$ a complex root there must exist 
$\sigma$ such that $\sigma(\beta) = \alpha$. Then, letting $\tau$ denote the automorphism of 
complex conjugation, if $\text{Aut}(E|\mathbb{Q})$ is Abelian it follows that $\sigma\tau(\beta) = \tau\sigma(\beta) = \alpha$. 
However since $\mathbb{Q}$ is of characteristic zero this violates the requirement that $\sigma$ be 
one to one as $\tau(\beta) \neq \beta$. Thus $\text{Aut}(E|\mathbb{Q}) = \text{Gal}(f)$ is not Abelian.

\paragraph{}
For an non-irreducible polynomial this is trivially false. Consider $f(x) = (x^2+1)(x-1)$. This 
polynomial has both real and complex roots, but the only element of $\text{Gal}(f)$ is 
complex conjugation. Therefore $\text{Gal}(f) \cong C_2$ which is Abelian.

\subsection*{4.3}
Let $E|F$ be a Galois extension of degree 2700, so $\#\text{Aut}(E|F) = 2700$. From the Sylow Theorems 
it follows that there must exist a subgroup $H$ where $\#H = 27$. Therefore $[\text{Aut}(E|F):H] = 1000$ and so 
the corresponding fixed field has degree $[E^H:F] = 1000$ as required.

\subsection*{4.4}
Let $G = \text{Aut}(E|F)$, where $E|F$ Galois and $\#G=625 = 5^4$. Therefore $G$ is a p-group and 
so there exists one subgroup of order $5^k$ for $k = \{1,2,3\}$. Consequently there are three intermediate 
fields $M_1$, $M_2$ and $M_3$ which have degrees $125$, $25$ and $5$ respectively.

\subsection*{4.5}
To find the subfields of $\mathbb{Q}(\eta)$ it will be useful to consider the structure of the 
Galois group. In particular we have 

\begin{equation*}
    \text{Aut}(\mathbb{Q}(\eta)|\mathbb{Q}) \cong \mathbb{Z}_{11}^\times \cong C_2 \times C_5
\end{equation*}

Where the last isomorphism follows from the Chinese Remainder Theorem. It is hopefully clear that 
there are two subgroups of this group, $H_2$ isomorphic to $C_2$ and $H_5$ isomorphic to $C_5$. 
One can generate $H_2$ by choosing any element mapped to an element of order 2 in 
$\mathbb{Z}_{11}^\times$ and since $10^2 \cong 1$ mod 11 it follows that $\sigma_{10}:\eta \rightarrow \eta^{10}$ 
generates $H_2$. By an identical argument we have $<\sigma_3> = H_5$, $\sigma_3:\eta \rightarrow \eta^3$. 
It will be useful later to note that this implies $<\sigma_3,\sigma_{10}> = \text{Aut}(\mathbb{Q}(\eta)|\mathbb{Q})$. 
The Galois correspondence then implies the following subfield diagram:

\begin{figure}[H]
    \centering
    \begin{tikzpicture}[invis/.style={rectangle, minimum size=0.5cm},node distance = 4cm]
        \node[invis]    (Q_eta)    {$\mathbb{Q}(\eta)$};
        \node[invis]    (Q_sigma10)     [below left of= Q_eta] {$\mathbb{Q}(\eta)^{<\sigma_{10}>}$};
        \node[invis]    (Q_sigma3)     [below right of= Q_eta] {$\mathbb{Q}(\eta)^{<\sigma_{3}>}$};
        \node[invis]    (Q)     [below right of= Q_sigma10] {$\mathbb{Q}$};

        \draw (Q_eta.south west) -- (Q_sigma10.north);
        \draw (Q_eta.south east) -- (Q_sigma3.north); 
        \draw (Q_sigma10.south) -- (Q.north west); 
        \draw (Q_sigma3.south) -- (Q.north east); 
    \end{tikzpicture}
\end{figure}

So there are two subfields of $\mathbb{Q}(\eta)$ over $\mathbb{Q}$, $\mathbb{Q}(\eta)^{<\sigma_{10}>}$ 
and $\mathbb{Q}(\eta)^{<\sigma_3>}$. It remains to find the minimal polynomial of a primitive 
element of each as extensions of $\mathbb{Q}$. 

\paragraph{}
Consider $\mathbb{Q}(\eta)^{<\sigma_{10}>}$. Since the map $\sigma_{10}$ takes $\eta$ to $\bar{\eta}$ 
clearly $\omega = \frac{\eta + \bar{\eta}}{2}$ is fixed under this automorphism. Consider then the orbit of 
$\omega$ under $\sigma_3$. It is easy to show that the orbit of $\omega$ is 

\begin{equation*}
    \mathcal{O}(\omega) = \left\{\frac{\eta + \bar{\eta}}{2},\frac{\eta^3 + \bar{\eta}^3}{2},\frac{\eta^2 + \bar{\eta}^2}{2},\frac{\eta^5 + \bar{\eta}^5}{2},\frac{\eta^4 + \bar{\eta}^4}{2}\right\}
\end{equation*}

Note that since these elements are not all equal it follows that $\omega$ is not in $\mathbb{Q}$. So the polynomial 

\begin{align*}
    f(x) &= \Pi_{\sigma \in <\sigma_3>}(x - \sigma(\omega)) = \left(x - \frac{\eta + \bar{\eta}}{2}\right)\left(x - \frac{\eta^2 + \bar{\eta}^2}{2}\right)\left(x - \frac{\eta^3 + \bar{\eta}^3}{2}\right)\left(x - \frac{\eta^4 + \bar{\eta}^4}{2}\right)\left(\frac{\eta^5 + \bar{\eta}^5}{2}\right) \\
    \implies f(x) & = x^5+\frac{1}{2}x^4-x^3-\frac{3}{8}x^2+\frac{3}{16}x+ \frac{1}{32}
\end{align*}

Is clearly fixed under every element in $\text{Aut}(\mathbb{Q}(\eta)|\mathbb{Q})$ and thus $f(x) \in \mathbb{Q}[x]$. 
The minimal polynomial of $\omega$ must thus divide $f(x)$.

\paragraph{}
However any polynomial $g(x) \in \mathbb{Q}[x]$ that 
divides $f(x)$ must include one root $\omega'$ of $f(x)$. Since this polynomial is in $\mathbb{Q}[x]$ 
it must be fixed under every automorphism in $\text{Aut}(\mathbb{Q}(\eta)|\mathbb{Q})$ and so 
every element in $\mathcal{O}(\omega')$ must be a root of $g(x)$. It is hopefully obvious that $\mathcal{O}(\omega') = \mathcal{O}(\omega)$ 
and thus the roots of $g(x)$ must include every element in $\mathcal{O}(\omega)$ (e.g. $f(x)|g(x)$). 

\paragraph{}
Therefore 
$g(x)|f(x) \implies g(x) = f(x)$ and so $f(x)$ is the minimal polynomial of $\omega$. Since 
$\mathbb{Q}(\eta)^{<\sigma_{10}>}$ has degree five from the Galois correspondence, $\mathbb{Q}(\omega)$ 
has degree five and the above clearly implies $\mathbb{Q}(\omega) \subseteq \mathbb{Q}(\eta)^{<\sigma_{10}>}$ 
we have that $\mathbb{Q}(\omega) = \mathbb{Q}(\eta)^{<\sigma_{10}>}$. Thus the minimal polynomial of a 
primitive element $\omega \in \mathbb{Q}(\eta)^{<\sigma_{10}>}$ is $f(x)$.

\paragraph{}
Considering the second subfield, $\mathbb{Q}(\eta)^{<\sigma_3>}$ it is easy to see that 
$\gamma = \eta + \eta^3 + \eta^4 + \eta^5 + \eta^9$ is fixed under $\sigma_3$ and that the orbit 
of $\gamma$ under $\text{Aut}(\mathbb{Q}(\eta)|\mathbb{Q})$ is simply 

\begin{equation*}
    \mathcal{O}(\gamma) = \left\{\eta + \eta^3 + \eta^4 + \eta^5 + \eta^9,\eta^2 + \eta^6 + \eta^7 + \eta^8 + \eta^{10}\right\} = \{\gamma,\bar{\gamma}\}
\end{equation*}

Note $\gamma \neq \bar{\gamma}$ and it also follows that the polynomial 

\begin{equation*}
    g(x) = (x-\gamma)(x-\bar{\gamma})
\end{equation*}

is fixed under all $\sigma \in \text{Aut}(\mathbb{Q}(\eta)|\mathbb{Q})$ and from an identical argument 
to that above is thus the minimal polynomial of $\mathbb{Q}(\gamma)$. Furthermore the Galois correspondence 
implies that $\mathbb{Q}(\eta)^{<\sigma_3>}$ is a quadratic extension and since $\mathbb{Q}(\gamma)$ is 
also of degree two we have $\mathbb{Q}(\eta)^{<\sigma_3>} = \mathbb{Q}(\gamma)$. Therefore $g(x)$ 
is the minimal polynomial of a primitive element $\gamma$ in $\mathbb{Q}(\eta)^{<\sigma_3>}$.

\subsection*{4.6}
Consider $g(x) = x^4+x^3+x^2+x+1$. Clearly this polynomial splits over $\mathbb{Q}(\eta)$ where 
$\eta^5 =1$, $\eta \neq 1$ and $\text{Gal}(g) \cong C_4$. Then let $h(x) = x^2 -2$, so 
$\text{Gal}(h) \cong C_2$. Finally, let $f(x) = g(x)h(x) \in \mathbb{Q}[x]$. It is clear that 
the splitting field of $f(x)$ will be $\mathbb{Q}(\eta,\sqrt{2})$. We claim that 
that $\text{Aut}(\mathbb{Q}(\eta,\sqrt{2})|\mathbb{Q}) \cong C_2 \times C_4$, so it remains to show this is the case. 
First, we wish to find the size of 
the associated automorphism group $\text{Aut}(\mathbb{Q}(\eta,\sqrt{2})|\mathbb{Q})$.

\paragraph{}
Obviously $[\mathbb{Q}(\sqrt{2}):\mathbb{Q}] = 2$. To find the degree of this extension we must 
also find $[\mathbb{Q}(\sqrt{2},\eta):\mathbb{Q}(\eta)]$. Since the minimal polynomial of $\eta$ in 
$\mathbb{Q}$ is $g(x)$, the minimal polynomial of $\eta$ in $\mathbb{Q}(\sqrt{2})$ must divide 
$g(x)$. Therefore it must be a product of the monic linear factors, $x - \eta^k$, $k \in \{1,2,3,4\}$. 
Since we require that all coefficients must be in $\mathbb{Q}(\sqrt{2})$ it must be the case that the 
constant term is $\eta^{5m}$ for some $m>0$. Suppose minimal polynomial is a quadratic. Then the 
above condition restricts our choice of polynomials to two possibilities

\begin{align*}
    (x- \eta)(x-\bar{\eta}) &= x^2 -\frac{1}{2}(\sqrt{5}-1)x + 1 \in \mathbb{Q}(\sqrt{5})[x] \\
    (x- \eta^2)(x-\bar{\eta}^2) &= x^2 + \frac{1}{2}(\sqrt{5}+1)x + 1 \in \mathbb{Q}(\sqrt{5})[x] \\
\end{align*}

Since $\mathbb{Q}(\sqrt{5}) \cap \mathbb{Q}(\sqrt{2}) = \mathbb{Q}$ these polynomials are not in $\mathbb{Q}(\sqrt{2})[x]$. Furthermore the 
divisor cannot be a cubic as the sum of any three elements from $\{1,2,3,4\}$ is not a divisible 
by 5. Therefore $g(x)$ is irreducible over $\mathbb{Q}(\sqrt{2})$ so it is the minimal polynomial 
of $\eta$ over this field. Thus $[\mathbb{Q}(\eta,\sqrt{2}):\mathbb{Q}(\sqrt{2})] = 4$. Therefore 
$[\mathbb{Q}(\eta,\sqrt{2}):\mathbb{Q}] = 8$ and so $\#\text{Aut}(\mathbb{Q}(\eta,\sqrt{2})|\mathbb{Q}) = 8$. 

\paragraph{}
Since $\mathbb{Q}(\eta)$ and $\mathbb{Q}(\sqrt{2})$ are subfields of the splitting field of $f$ 
there are obviously two subgroups of $H_\eta, H_{\sqrt{2}} <\text{Aut}(\mathbb{Q}(\eta,\sqrt{2})|\mathbb{Q}(\sqrt{2}))$ 
where $H_\eta$ fixes $\mathbb{Q}(\eta)$ and $\mathbb{Q}(\sqrt{2})$ is fixed by $H_{\sqrt{2}}$. 
Since $\mathbb{Q}(\eta)$ is the splitting field of $g(x)$ which is irreducible in $\mathbb{Q}[x]$ 
and $\mathbb{Q}(\sqrt{2})$ is the splitting field of $h(x)$ which is also irreducible in $\mathbb{Q}[x]$ 
the Galois correspondence implies both subgroups are normal. In addition  
$\phi \in H_{\sqrt{2}}$ fixes $\sqrt{2}$ and permutes roots of $g(x)$, so it is hopefully obvious 
that valid automorphisms will send $\eta$ to powers of $\eta$ and that $\phi:\eta \rightarrow \eta^2$ 
is sufficient to generate the whole group of automorphisms which must be isomorphic to $C_4$. 

\paragraph{}
Clearly $H_\eta$ acts only on the roots of $h(x)$ and has size two. Therefore the only non-identity 
element must send $\sqrt{2} \rightarrow -\sqrt{2}$. Thus $H_\eta \cong C_2$. Since $H_\eta$ and $H_{\sqrt{2}}$ have trivial intersection (as 
an automorphism which fixes both $\eta$ and $\sqrt{2}$ must clearly be the identity automorphism)
and $\#H_\eta\#H_{\sqrt{2}} = \#\text{Aut}(\mathbb{Q}(\eta,\sqrt{2})|\mathbb{Q})$ it follows 
that $\text{Aut}(\mathbb{Q}(\eta,\sqrt{2})|\mathbb{Q}) \cong H_{\sqrt{2}} \times H_\eta \cong C_2 \times C_4$ as required.

\subsection*{4.7}
Consider $x^{256}-x$. This polynomial has 256 roots, corresponding to each element of $\mathbb{F}_{256}$. 
If $g(x)$ is an irreducible monic factor of $x^{256}-x$ it clearly must be of the form 

\begin{equation*}
    g(x) = (x-\alpha_1)(x-\alpha_2)\dots(x-\alpha_n),\qquad \alpha_i \in \mathbb{F}_{256},\quad \alpha_i \neq \alpha_j\quad \forall i \neq j
\end{equation*}

In particular, $g(x) \in \mathbb{F}_2[x]$ must be of the form 

\begin{equation*}
    g(x) = \Pi_{\alpha_i \in \mathcal{O}(\alpha)}(x - \alpha_i)
\end{equation*}

For some $\alpha \in \mathbb{F}_{256}$. Otherwise, if $\alpha_* \in \mathcal{O}(\alpha)$ and 
$\alpha_*$ is not a root of the polynomial then there exists an automorphism which sends one root 
of the polynomial to $\alpha_*$ so the resulting polynomial is not fixed under this automorphism 
and thus not in $\mathbb{F}_2[x]$. The above also implies that if $\beta \notin \mathcal{O}(\alpha)$ is 
a root of $g(x)$ then the entire orbit of $\beta$ must also be included so $g(x)$ can be written as 

\begin{equation*}
    g(x) = h_\alpha(x)h_\beta(x) = \Pi_{\alpha_i \in \mathcal{O}(\alpha)}(x - \alpha_i)\Pi_{\beta_i \in \mathcal{O}(\beta)}(x - \beta_i)
\end{equation*}

and this polynomial is not irreducible as both $h_\alpha(x)$ and $h_\beta(x)$ are in $\mathbb{F}_2[x]$. 
Since the set of orbits of elements in $\mathbb{F}_{256}$ under $\text{Aut}(\mathbb{F}_{256}|\mathbb{F}_2)$ 
forms a partition of $\mathbb{F}_{256}$ the number of monic irreducible factors is thus the number 
of distinct orbits in $\mathbb{F}_{256}$.

\paragraph{}
Next it is useful to consider the possible size of orbits for elements in each of the subfields of 
$\mathbb{F}_{256}$. If $\alpha \in \mathbb{F}_2$ then clearly $\#\mathcal{O}(\alpha) = 1$ as this 
field is fixed by the definition of the automorphism group. Elements in $\mathbb{F}_4\backslash \mathbb{F}_2$ are in a 
degree two extension, so the maximal subgroup that fixes these is elements is a subgroup of index two in $\text{Aut}(\mathbb{F}_{256}|\mathbb{F}_2)$. 
Since $\text{Aut}(\mathbb{F}_{256}|\mathbb{F}_2)$ has size eight, this implies the subgroup has size 
four. Therefore the stabilizer of $\alpha \in \mathbb{F}_4\backslash \mathbb{F}_2$ has size four and the orbit has size two. 
By an identical argument the stabilizer of $\alpha \in \mathbb{F}_{16}\backslash \mathbb{F}_4$ has size 
two and so $\#\mathcal{O}(\alpha) = 4$. Finally if $\alpha \in \mathbb{F}_{256}\backslash \mathbb{F}_{16}$ 
the stabilizer is trivial and $\#\mathcal{O}(\alpha) = 8$. 

\paragraph{}
Since there are two elements in $\mathbb{F}_2$ both these elements have an orbit of size one. 
There are two elements in $\mathbb{F}_4\backslash \mathbb{F}_2$ and each of these elements belongs 
in an orbit of size two, so there is one orbit of size two. There are twelve elements in $\mathbb{F}_{16}\backslash \mathbb{F}_4$ 
and each of these belong in an orbit of size four, so there are three orbits of size four. Finally 
there are 240 elements in $\mathbb{F}_{256}\backslash \mathbb{F}_{16}$ each of which belong to 
orbits of size eight so there are 30 orbits of size eight. 

\paragraph{}
Consequently there are 36 monic irreducible factors of $x^{256}-x$. However, one of these is 
$x$ which is clearly not a factor of $x^{255}-1 = (x^{256}-x)/x$. Consequently there are 
35 irreducible monic factors of $x^{255}-1$, one of which is linear, one is quadratic, three are quartic and 
30 have degree eight.


\subsection*{4.8}
If $f(x) \in \mathbb{Q}[x]$ and has degree less than 4, then $\text{Gal}(f)$ is a subgroup of 
$S_4$. Galois' theorem states that $f$ is solvable in radicals if and only if $G = \text{Gal}(f)$ 
is solvable. Since $G$ is a subgroup of $S_4$, its order must divide $S_4$ and thus $\#G = 2^n3^k$ 
for $n \in \{0,1,2,3\}$, $k = \{0,1\}$. Clearly if $n$ and/or $k$ equal zero then the resulting 
group is either a two group or a three group. All subgroups of such groups are normal and for 
any two or three group each of order $2^r$ or $3^r$ there exists subgroups of that group for each $r-1,r-2,\dots,0$. 
This implies the existence of a Jordan-H{\"o}lder series where each Jordan-H{\"o}lder factor is 
either $C_2$ or $C_3$ and consequently they must be solvable.

\paragraph{}
Clearly if both $n$ and $k$ are greater than zero, then $G$ must have order 6, 12 or 24 and so it is hopefully obvious that $G$ must be 
one of $S_3$, $A_4$ or $S_4$. In assignment 2 we demonstrated that $S_4$ was solvable and since 
this composition series passed through $A_4$ this is sufficient to demonstrate that $A_4$ is 
also solvable. Finally $S_3 \triangleright A_3 \triangleright \{1\}$ and it is clear from 
this that the Jordan-H{\"o}lder series contains only $C_3$ and $C_2$ so $S_3$ is solvable. Thus all polynomials in $\mathbb{Q}[x]$ of degree four or less are 
solvable in radicals.

\end{document}