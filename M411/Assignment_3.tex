\documentclass{article}

\title{	
	\normalfont\normalsize 
	\rule{\linewidth}{0.5pt}\\ % Thin top horizontal rule
	\vspace{14pt} % Whitespace
	{\LARGE MATH411 Assignment 2\\ % The assignment title
    \large \textit{} \\}
	\vspace{6pt} % Whitespace
	\rule{\linewidth}{1pt}\\ % Thick bottom horizontal rule
}

\author{Elliott Hughes}
\date{\normalsize\today}
\usepackage{tikz}
\usetikzlibrary{arrows,automata}
\usetikzlibrary{positioning}
\usetikzlibrary{arrows.meta,positioning}
\usepackage{mdframed}
\usepackage{amsmath}
\usepackage{amssymb}
\usepackage{graphicx}
\graphicspath{ {./Images/} }
\usepackage{commath}
\usepackage{textcomp}
\usepackage{gensymb}
\usepackage{float}
\usepackage{hyperref}
\usepackage[margin=1in]{geometry}
\usepackage{caption}
\usepackage{subcaption}
\usepackage{sectsty}
\usepackage{titlesec}
\def\acts{\rotatebox[origin=c]{-60}{$\circlearrowright$}}

\begin{document}

\maketitle

\subsection*{3.1}
Let $E|F$ be a Galois extension of a field $F$ and let $f(x) \in F[x]$ irreducible over $F$. Then 
$f(x)$ can be factored into a product of some scalar and a set of monic polynomials irreducible 
over $E|F$, $f(x) = af_1(x)f_2(x)\dots f_n(x)$, $a \in F$. For some $\sigma \in \text{Gal}(E|F)$ it follows that 

\begin{equation*}
	a\sigma(f_1(x))\sigma(f_2(x))\dots \sigma(f_n(x)) = af_1(x)f_2(x)\dots f_n(x)
\end{equation*}

Since $f$ has a unique factorization up to order in $E$ and $\sigma$ must map monic polynomials to 
monic polynomials it follows that $\sigma(f_i) = f_j$ for some $f_i$, $f_j$ in the factorization. 
It remains to show that this groups acts transitively on these irreducible factors. 

\paragraph{}
Consider the action $\text{Aut}(E|F) \acts \{f_j\}$, where $\{f_j\}$ is the set of 
irreducible factors. Then for $f_j$ in this set, denote the product of irreducible factors in 
this orbit as $p(x)$. Cearly $p(x)|f(x)$ and, furthermore, note that for $\phi \in \text{Aut}(E|F)$, 
$\phi(p(x)) = p(x)$ by the definition of orbits. Since this implies that each coefficient in $p(x)$ 
is fixed for all $\phi \in \text{Aut}(E|F)$ it follows from the fact that $E|F$ is Galois that 
$p(x) \in F[x]$. Then since $f(x)$ is irreducible this implies that $f(x) = ap(x)$ and therefore 
that $\text{Aut}(E|F)$ acts transitively on the irreducible factors of $f(x)$ in $E$.

\subsection*{3.2}
Let $E|F$ be an extension of $F$ and let $H$ be a subgroup of $\text{Aut}(E|F)$. The set $E^H$ is 
non-empty as clearly $F \subseteq E^H$. Then for $a,b \in E^H$ and $\phi \in H$ we have 
$\phi(a - b) = \phi(a) -\phi(b) = a -b$ so $a-b \in E^H$. Furthermore, for $a \in E^H$, $b \in E^H\backslash\{0\}$ 
we have $\phi(ab^{-1}) = \phi(a)\phi(b^{-1}) = \phi(a)(\phi(b))^{-1} = ab^{-1}$ so $E^H$ is a subfield 
of $E|F$ and consequently a field.

\subsection*{3.3}
Let $E|F$ be an algebraic extension, $\theta \in E$ and $g(x) \in F[x]$ the minimal polynomial of 
$\theta$ over $F$. For $\sigma \in \text{Aut}(E|F)$ we have $\sigma(g(\theta)) = \sigma(g)(\sigma(\theta)) = g(\sigma(\theta)) = \sigma(0) = 0$ 
and so $\sigma(\theta)$ is a root of $g$.

\paragraph{}
For $f \in F[x]$, $E|F$ splitting $f$ it follows that there exists a finite set of roots in 
$E|F$, $R = \{\theta_0,\theta_1,\dots,\theta_n\}$. Since for any $\sigma \in \text{Gal}(f)$ and for 
any $\theta \in R$ it must be true that $\sigma(\theta) \in R$ we can define the action $\sigma \star \theta = \sigma(\theta)$. 
It remains to show that this fulfills the definition of a group action. Since the identity automorphism 
is clearly the identity of $\text{Gal}(f)$ the first requirement is straightforwardly fulfilled. 
The second restriction that for $\sigma_1,\sigma_2 \in \text{Gal}(E|F)$ and $\theta \in R$, 
$\sigma_1 \star \sigma_2 \star \theta = \sigma_1\sigma_2 \star \theta$ is clearly fulfilled by the 
properties of automorphism groups. Therefore this action is a valid group action.

\subsection*{3.4}
Let $f \in F[x]$ and $E|F$ be the splitting field of $f$, with the set of roots $R = \{\theta_1,\theta_2,\dots,\theta_n\}$. 
From 3.3 and the requirement that for any $\sigma \in \text{Aut}(E|F)$, $\sigma(f) = f$ it 
follows that each $\sigma$ defines a permutation on $R$. Therefore $\text{Aut}(E|F) \cong H \leq \text{Sym}(R)$. 

\paragraph{}
Assume that this homomorphism is not injective. This implies that the kernel is non-trivial and thus that 
there exists a non identity $\sigma \in \text{Aut}(E|F)$ such that $\sigma(\theta) = \theta$ for all 
$\theta \in R$. However since $E = F(\theta_1,\theta_2,\dots,\theta_n)$, if $\sigma(\theta) = \theta$ for all $\theta \in R$ then since 
every element can be written as a sum and/or product of elements fixed by $\sigma$ it follows that 
$\sigma(x) = x$ for all $x \in E$ which contradicts our assumption. Therefore this homomorphism is injective.

\subsection*{3.5}
Clearly, $\phi(\theta) = \phi(\eta^3\theta) = \eta\theta$ and $\phi(\eta\theta) = \eta^2\theta$ 
which implies $\phi(\eta) = \phi(\eta\theta\theta^{-1}) = \phi(\eta\theta)\phi(\theta^{-1}) = \eta^2\theta(\eta\theta)^{-1} = \eta$. From the actions on these two 
elements and the fact that $\phi$ fixes any element in $F$ we can construct the matrix that gives 
the action of $\phi$ on elements of the vector space of $E$ over $\mathbb{Q}$.

\begin{equation*}
	\begin{bmatrix}
		1 & 0 & 0 & 0 & 0 & 0\\
		0 & 0 & 0 & 0 & 1 & 0\\
		0 & 0 & 1 & 0 & 0 & 1\\
		0 & 0 & 0 & 1 & 0 & 0\\
		0 & 1 & 0 & 0 & -1 & 0\\
		0 & 0 & -1 & 0 & 0 & 0 
	\end{bmatrix}
\end{equation*}

The corresponding 1-eigenspace is comprised of the first and fourth columns, or $\mathbb{Q}(\eta)$. This 
agrees with our initial computation that $\phi(\eta) = \eta$.

\subsection*{3.6}
Clearly the roots of this polynomial are products of $\eta$, $\zeta$ where $\eta$ is one of the 
fifth roots of unity and $\zeta = 2^{1/5}$. The splitting field can then be straightforwardly written 
as $\mathbb{Q}(\eta,\zeta)$. To compute the degree of this extension, it is convenient to 
consider the degrees of constituent subfields. In particular, consider $\mathbb{Q}(\eta)$. 
This is the cyclotomic extension of $\mathbb{Q}$ corresponding to the fifth root of 
unity. Therefore $[\mathbb{Q}(\eta):\mathbb{Q}] = 4$ by theorem 5.1. 

\paragraph{}
Consider $\mathbb{Q}(\eta,\zeta)|\mathbb{Q}(\eta)$. This extension splits $x^5 -2$ and every root of 
this polynomial contains $\zeta$, so any divisor of this polynomial will have a constant term 
corresponding to some power of $\zeta$ (potentially multiplied by some other term). Then $\zeta^n \notin \mathbb{Q}(\eta)$ for $n \in \{1,2,3,4\}$ 
so this is the minimal polynomial over $\mathbb{Q}(\eta)$. Consequently 
$[\mathbb{Q}(\eta,\zeta):\mathbb{Q}(\eta)] = 5$. Therefore the degree of the splitting field of 
$x^5 -2$ is $[\mathbb{Q}(\eta,\zeta):\mathbb{Q}] = 20$ by the Tower Theorem.

\paragraph{}
Consequently $\text{Aut}(\mathbb{Q}(\eta,\zeta)|\mathbb{Q}) \cong H < S_5$ with $|H| = 20$. A subgroup 
of order 20 must contain $s_2$ Sylow-2-Groups and $s_5$ Sylow-5-Groups. In particular $s_5 \cong 1$ 
mod 5 and $s_5 | 2$ so $s_5 = 1$ 
so $H$ contains a single group of order 5.

\paragraph{}
Furthermore, consider the subgroup $K < \text{Aut}(\mathbb{Q}(\eta,\zeta)|\mathbb{Q})$, where 
$\phi(\zeta) = \zeta$. This is a subgroup of the automorphism group and it is clearly isomorphic 
to $\text{Aut}(\mathbb{Q}(\eta)|\mathbb{Q}) \cong \mathbb{Z}_4^+$. Therefore there is a subgroup of the 
automorphism group (we will denote this subgroup $A$) and $A \cong \mathbb{Z}_4^+$. Clearly then if $a \in A$ 
and $b$ is an element of the Sylow-5-Group then $H$ is generated by $a$ and $b$. Since a 
group of order 10 cannot contain elements of order four and it must contain $b$ it follows that 
such a group must be generated by $a^2$ and $b$. Furthermore all elements in the Sylow-5-Group 
are conjugate to some other element in that group, $a^2b^r = b^na^2$ for some $n,r \in \mathbb{N}$. 
The set generated by $a^2$ and $b$ is therefore a subgroup of $H$ and since it is the only 
possible such subgroup of order 10 it follows that there is one subgroup of index 2 in $\text{Aut}(\mathbb{Q}(\eta,\zeta)|\mathbb{Q})$. 
The Galois correspondence then implies that there is only one quadratic subfield in $\mathbb{Q}(\eta,\zeta)|\mathbb{Q}$.

\subsection*{3.7}
The homomorphism $\Phi:\text{Norm}(H) \rightarrow \text{Aut}(E^H|F)$ (where $\text{Norm}(H)$ is 
the normalizer of $H$) induced by restricting the 
action of $\phi \in \text{Norm}(H)$ to $E^H$ is well-defined as $\phi(E^H) = E^H$ by the third 
property of the Galois Correspondence. Then clearly $\text{ker}(\Phi) = H$ and so there is an 
injective homomorphism between $\text{Norm}(H)/H$ and $\text{Aut}(E|F)$.

\paragraph{}
Consider $\phi$ an injective homomorphism from a subfield $K$ to $E$ and write 
$E = K(\alpha_1,\alpha_2,\dots,\alpha_l)$ 
(note that $E$ is finite so the set of elements that must be adjoined is finite). 
Clearly $E$ is algebraic over $K$ as $E$ is the splitting field of some polynomial in 
$f(x) \in F[x] \subset K[x]$. Thus for $\alpha_1$ the minimal polynomial of $\alpha_1$ in 
$K$ exists, is irreducible and has at least one root in $E$, so there is at least one extension of 
$\phi$ which is an injective homomorphism from $K(\alpha_1)$ to $E$ that agrees on $K$ by Lemma 3.4.

\paragraph{}
Let $K_0 = E^H$. Then from above we can write $E = K_0(\alpha_1,\alpha_2,\dots,\alpha_n)$.
Consider a sequence of subfields $K_i$, $i = 1,2,\dots,n$ with $K_1 = E^H(\alpha_1)$, $K_{i+1} = K_i(\alpha_{i+1})$. 
Then for $\phi \in \text{Aut}(E^H|F)$ define $\phi_1$ as the injective homomorphism from 
$K_1 \rightarrow E$ which agrees with $\phi$ on $E^H$. From the argument above, we can inductively define a 
sequence of injective homomorphisms from $K_i \rightarrow E$ which agree with $\phi$ on $E^H$. 
In particular there exists $\phi_n$ an injective isomorphism from $K_n = E$ to $E$ which agrees 
with $\phi$ on $E^H$.

\paragraph{}
We wish to show that this is an automorphism of $E|F$, so it remains to show that it is surjective. 
Since $\phi_n$ agrees with $\phi$ on $E^H$ it follows that $\phi_n(0) = 0$. Then for $a \in E$, 
$a \neq 0$ and $\phi_n(a) \neq 0$ as $\phi_n$ is injective. Furthermore if $\phi_n(a) = b$ then 
the injectivity of $\phi_n$ implies $\phi_n^{-1}(b) = a$ and so $\phi_n(b^{-1}) = a$. Since 
$b^{-1}$ is well-defined it follows that $\phi_n$ is surjective and so $\phi_n$ is an automorphism 
($F$ is trivially fixed as $\phi_n$ agrees with $\phi \in \text{Aut}(E^H|F)$ on $F$). 

\paragraph{}
Thus for $\phi \in \text{Aut}(E|F)$ and the map $\phi \rightarrow \phi_n$ is 
clearly injective as the action of any two elements of $\text{Aut}(E^H|F)$ differs on $E^H$ 
so it will differ on $E$. Trivially $\phi(E^H) = E^H$ and so $\phi_n \in \text{Norm}(H)$. 
As their action differs on $E^H$ each $\phi_n$ belongs to a different coset of $\text{Norm}(H)/H$ 
and so there is an injective homomorphism between $\text{Aut}(E|F)$ and $\text{Norm}(H)/H$. Therefore 
these groups are isomorphic.


\end{document}