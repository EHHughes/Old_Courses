\documentclass{article}

\title{	
	\normalfont\normalsize 
	\rule{\linewidth}{0.5pt}\\ % Thin top horizontal rule
	\vspace{14pt} % Whitespace
	{\LARGE MATH411 Assignment 1\\ % The assignment title
    \large \textit{} \\}
	\vspace{6pt} % Whitespace
	\rule{\linewidth}{1pt}\\ % Thick bottom horizontal rule
}

\author{Elliott Hughes}
\date{\normalsize\today}
\usepackage{tikz}
\usetikzlibrary{arrows,automata}
\usetikzlibrary{positioning}
\usetikzlibrary{arrows.meta,positioning}
\usepackage{mdframed}
\usepackage{amsmath}
\usepackage{amssymb}
\usepackage{graphicx}
\graphicspath{ {./Images/} }
\usepackage{commath}
\usepackage{textcomp}
\usepackage{gensymb}
\usepackage{float}
\usepackage{hyperref}
\usepackage[margin=1in]{geometry}
\usepackage{caption}
\usepackage{subcaption}
\usepackage{sectsty}
\usepackage{titlesec}
\usepackage[backend=biber]{biblatex}
\addbibresource{Assignment_1.bib}

\begin{document}

\maketitle

\subsection*{1.1.1}
Let $g,h \in G'$ so by the properties of homomorphisms we have $\phi(g)\phi(h) = \phi(gh)$ and 
$gh \in G'$ since $G' < G$ (thus $\phi(G)$ is closed in $H$). Furthermore, if $g \in G'$ and $1$ is 
the identity element of $G$ then $\phi(g)\phi(1) = \phi(g)$ which 
implies that $\phi(1)$ is the identity of $H$ so $\phi(G')$ contains the identity.
If $g \in G'$ then $g^{-1} \in G$ and $\phi(g)\phi(g^{-1}) = \phi(gg^{-1}) = \phi(1)$ so 
for $\phi(g) \in \phi(G')$ we have $\phi(g^{-1}) = \phi(g)^{-1} \in \phi(G')$ and thus inverses exist 
for each element in $\phi(G')$. Finally we inherit the associative property since $H$ is a group. 
Thus $\phi(G') < H$.

\subsection*{1.1.2}
Consider $g,h \in G$. Then if $\phi(g), \phi(h) \in H'$ it follows that $\phi(gh) \in H'$ since 
$H' < H$. Thus $\forall g,h \in \phi^{-1}(H')$ it follows that $gh \in \phi^{-1}(H')$. Furthermore 
if $g \in G$ s.t. $\phi(g) \in H'$ then $\phi(g)^{-1} = \phi(g^{-1}) \in H'$ since $H' < H$ and 
consequently if $g \in \phi^{-1}(H')$ then $g^{-1} \in \phi^{-1}(H')$. Combining the two 
proceeding results
trivially implies that $1 \in \phi^{-1}(H')$ and the associativity property is inherited from the 
larger group $G$. Therefore $\phi^{-1}(H') < G$.

\subsection*{1.1.3}
From 1.9 (see below) there is a homomorphism from $S_3$ to some two element group (e.g. $\{1,\,(12)\}$) in $S_4$. Since 
$S_3  \triangleleft S_3$ but $(13)(12)(13) = (23) \notin \{1,\,(12)\}$, it follows that, in general, 
if $G' \triangleleft G$ then $\phi(G') < H$ but $\phi(G')$ may not be normal.

\subsection*{1.1.4}
Consider the homomorphism $\varphi: H \rightarrow H/H'$ defined by $\varphi(g) = gH'$. This is a 
homomorphism as $H'$ is normal and it has kernel $H'$. It follows therefore that $\varphi \circ \phi$ 
is also a homomorphism with $ker(\varphi \circ \phi) = \phi^{-1}(H')$ so $\phi^{-1}(H')$ is normal 
by lemma 3.10 (in J.S. Milne's notes).

\subsection*{1.2.1}
For $h,n \in H \cap N$ we have that $hn \in H$ and $hn \in N$ so $hn \in H \cap N$. Then for 
$h \in H \cap N$ we have $h^{-1} \in H$ as $H < G$ and $h^{-1} \in N$ by an identical argument. 
Thus if $h \in H \cap N$ it follows that $h^{-1} \in H \cap N$ and as an immediate consequence that 
$1 \in H \cap N$. Thus $H\cap N < G \implies H \cap N < H$ (as $H \cap N \subset H$ trivially). 

\paragraph{}
It remains to show that $H \cap N \triangleleft H$. Note that since $N$ is normal, for 
$h,h^{-1} \in H$ and $n \in H \cap N$ we have that $hnh^{-1} \in N$ and since $h,h^{-1},n \in H$ 
we also have that $hnh^{-1} \in H \cap N$. Thus $H \cap N$ is a normal subgroup of $H$.

\subsection*{1.2.2}
As above it is clear that $H \cap N < G$. Consider the group $ G = Sym(3)$ with $N = Sym(3)$ and 
$H = \{1, (12)\}$. Then clearly $H \cap N = \{1, (12)\}$ but this group is not a normal 
subgroup (e.g.$(13) \in G$ but $(13)(12)(31) = (23) \notin H$). Thus in general it is not the case that $H \cap N \triangleleft G$.

\subsection*{1.2.3}
Note that $HN$ is a subgroup as for any $h_1n_1,h_2n_2 \in HN$ we have that $h_1n_1h_2n_2 = h_1n_1n_3h_2 = h_1h_2n_4$ 
for some $n_3,n_4 \in N$ as $N$ is normal. Thus $HN$ is closed, inverses exist as for some $h_1n_1 \in HN$ we have that 
$(h_1n_1)^{-1} = n_1^{-1}h^{-1} = h^{-1}n_2$ for some $n_2 \in N$ (as $N$ is normal). These first two properties together 
imply that $1 \in HN$ and so $HN < G$. 

\paragraph{}
It is further straightforward to show that this is not a normal subgroup by considering $Sym(3)$ with 
$N = \{1\}$ and $H = \{1,(12)\}$. Clearly $HN = \{1,(12)\}$ which is not normal as previously discussed in 1.2.2.

\subsection*{1.3}
For $n = n_1 + n_2 + \dots + n_r$ we let $N = Sym(n)$. Furthermore, let $H_1 = Sym(n_1)$ 
be the permutations of $1,2,\dots,n_1$, $H_2 = Sym(n_2)$ be the permutations of $n_1+1,n_1+2,\dots,n_1+n_2$, 
etc for all $H_1, \dots, H_r$. Then $H_i$ is a group and since $H \subset G$ we have that $H_i < G$.

\paragraph{}
Furthermore $H_1$, $H_2$ have no elements in common except the identity, so $H_1H_2$ must have 
$|H_1|\times|H_2|$ elements. Furthermore it is fairly straightforward to see that if $H_1H_2\dots H_k$ has $|H_1|\times |H_2| \times \dots \times |H_k|$ elements 
then $H_1H_2\dots H_k$ has 
no elements in common with $H_{k+1}$ except the identity so $H_1H_2\dots H_kH_{k+1}$ has 
$|H_1|\times |H_2| \times \dots |H_k|\times |H_{k+1}|$ elements. By induction we can see that 
$H_1H_2\dots H_n < G$ must have $\prod_{i=1}^r(n_i!)$ elements and that it must be a subgroup as each 
constituent group has no elements in common with any of the other groups except the identity.
Therefore $\prod_{i=1}^r(n_i!)$ divides $n!$ by Lagrange's Theorem.

\subsection*{1.4}
Note that the number of right and left cosets are the same (so $H$ must have two right and left cosets), so if $gh \in gH$ with $g \in G$, $h \in H$ 
then either $gh \in H$ which it is clearly not (as cosets disjointly partition the group $G$) or 
$gh \in Hg$. Therefore $gH = Hg$ and thus $H$ is normal.

\subsection*{1.5}
We have that $(\pm 1)(\pm 1) = \pm 1$, $(\pm 1)(\pm i) = \pm i$ and $(\pm i)(\pm i) = \pm 1$ so 
$\mu_4$ is closed. Furthermore it contains the identity and $(-i)(i) = 1 = -1(-1)$ so inverses 
exist. Thus $\mu_4$ is a subgroup. Then since $\mathbb{C}^{\times}$ is Abelian it follows that $\mu_4 \triangleleft \mathbb{C}^{\times}$.

\paragraph{}
We can define the function $\phi: \mathbb{C^\times}/\mu_4 \rightarrow \mathbb{C}^\times$ with 
$z\mu_4 = z^4$. Note that since for $u \in \mu_4$, $u^4 = 1$, this mapping is not dependent on the 
choice of coset leader.
This is a homomorphism as for $w_1,w_2 \in \mathbb{C}^\times$ then $\phi(w_1\mu_4)\phi(w_2\mu_4) = w_1^4w_2^4 = (w_1w_2)^4 = \phi(w_1w_2\mu_4)$. It is also 
injective as if $\phi(z) = \phi(w) \implies r_z^4 = r_w^4$ and that $\theta_z = \theta_w$ (mod($\pi/2)$) 
for $z = r_ze^{i\theta_z}$, $w = r_we^{i\theta_w}$ so $z,w \in z\mu_4$. Finally it is surjective 
as for $re^{i\theta} \in \mathbb{C}^\times$ it is clear that there exists $r^{1/4}e^{i\theta/4}$ 
which is an element of $r^{1/4}e^{i\theta/4}\mu_4$. Therefore it follows that $\phi$ is an 
isomorphism and so $\mathbb{C}^\times/\mu_4 \cong \mathbb{C}^\times$.

\paragraph{}
$\{\pm 1\}$ is a subgroup of $\mathbb{R}^\times$ as there exists a homomorphism $\phi:\mathbb{R}^\times \rightarrow \mathbb{R}^\times_+$ 
(where $\mathbb{R}^\times_+$ denotes the group of positive real numbers under multiplication) 
$\phi(x) = x^2$. This is a homomorphism as for $x,x' \in \mathbb{R}^\times$ we have $\phi(xx') = x^2x'^2 = \phi(x)\phi(x')$. 
Then $ker(\phi) = \{\pm 1\}$ so this must be a normal subgroup by lemma 3.10.

\paragraph{}
It is useful to notice that $\mathbb{R}^\times/{\pm 1}$ is isomorphic to $\mathbb{R}^\times_+$. The 
map $\phi: x\{\pm 1\} \rightarrow |x|$ is well-defined as clearly $\phi$ maps a coset to the same element 
regardless of the choice of coset leader. Furthermore, it is a homomorphism as 
for $x,y \in \mathbb{R}^\times$ we have $\phi(xy\{\pm 1\}) = |xy| = |x||y| = \phi(x\{\pm 1\})\phi(y\{\pm 1\})$. 
It is injective as if $\phi(x\{\pm 1\}) = \phi(y\{\pm 1\}) \implies |x| = |y| \implies x = \pm y \implies x,y \in x\{\pm 1\}$. 
It is also surjective as for $x \in \mathbb{R}^\times_+$ we have $\phi(x\{\pm 1\}) = |x| = x$. Therefore 
these two groups are isomorphic.

\paragraph{}
Thus the problem can be reduced to the somewhat simpler question: is $\mathbb{R}^\times_+$ isomorphic 
to $\mathbb{R}^\times$? If an isomorphism $\varphi$ from $\mathbb{R}^\times$ to $\mathbb{R}^\times_+$ does exist, this implies that $\varphi(1) = 1$ 
by the properties of isomorphisms. Therefore we have that $\varphi((-1)(-1)) = 1$ and so 
$\varphi(-1)\varphi(-1) = 1$. It follows that $\varphi(-1)^{-1} = \varphi(-1)$ and so this 
element of $\mathbb{R}^\times_+$ is its own inverse. But the only element in $\mathbb{R}^\times_+$ mapped to 
itself by $x \rightarrow x^2$ is $x = 1$. Therefore $\varphi$ cannot be an isomorphism as it is 
not injective and so we obtain a contradiction. Thus it follows that $\mathbb{R}^\times/\{\pm 1\}$ is 
not isomorphic to $\mathbb{R}^\times$.



\subsection*{1.6}
Let $G$ be a group and $H,N < G$. Then for $h_1,h_2 \in H$ and $g \in G$ we have 

\begin{equation*}
	gh_1g^{-1}gh_2g^{-1} = gh_1h_2g^{-1} \in gHg^{-1}
\end{equation*}

and since $h_1^{-1} \in H$ it follows that $gh_1^{-1}g^{-1} \in gHg^{-1}$ and therefore 
$gh_1g^{-1}gh_1^{-1}g^{-1} = gg^{-1} = 1$. Thus $gHg^{-1}$ is a subgroup. 

\paragraph{}
Define the following relation $H \sim N$ iff $H = gNg^{-1}$ for some $g \in G$. 
This is reflexive as $H = 1H1^{-1} = H$, symmetric as if $H = gNg^{-1}$ then $g^{-1}H = Ng^{-1}$ 
and thus $g^{-1}H(g^{-1})^{-1} = N$. It is also transitive as for $H_1,H_2,H_3 < G$ with 
$H_1 \sim H_2$ and $H_2 \sim H_3$ we have for some $g_1, g_2 \in G$

\begin{equation*}
	H_1 = g_1H_2g_1^{-1}, \qquad H_2 = g_2H_3g_2^{-1} \implies H_1 = g_1g_2H_3g_2^{-1}g_1^{-1} = (g_1g_2)H_3(g_1g_2)^{-1}
\end{equation*}

So $\sim$ is a valid equivalence relation.

\paragraph{}
Considering the group $Sym(3)$ there are the following subgroups $\{1\}$, $\{1,(13)\}$, $\{1,(23)\}$, $\{1, (12)\}$,
$\{1, (123), (132)\}$ and $Sym(3)$. Clearly the trivial subgroup and the group itself will not be conjugate to any other subgroups, as 
$g1g^{-1} = 1$ for all $g \in Sym(3)$. For $\{1,(13)\}$ we have that $(23)(13)(23) = (12)$ and that 
$(13)(12)(13) = (23)$ so $\{1, (12)\} \sim \{1,(13)\} \sim \{1, (23)\}$ by the properties of 
equivalence relations. Therefore each belong to the same equivalence class.

\paragraph{}
Finally for 
$\{1, (123), (132)\}$ we have that for each of $(12)$, $(13)$, $(23)$, the conjugacy operator 
returns the original subgroup $\{1, (123), (132)\}$. 

\begin{align*}
	(12)(123)(12) &= (132) \\
	(12)(132)(12) &= (123) \\
	(13)(123)(13) &= (132) \\
	(13)(132)(13) &= (123) \\
	(23)(123)(23) &= (132) \\
	(23)(132)(23) &= (123)
\end{align*}


Therefore this subgroup belongs in its own 
equivalence class. So the equivalence classes of $Sym(3)$ are the trivial group, the groups of 
order two, the group of order three and the group itself.

\subsection*{1.7}
The group $GL_n(F_q)$ is the set of all invertible $n \times n$ matrices with coefficients 
drawn from $F_q$ under matrix multiplication, which is equivalent to the set of all $n \times n$ 
matrices with coefficients drawn from $F_q$ that have full rank. This is equivalent to the 
statement that all columns are linearly independent in $F_q$, so we can construct a matrix in 
this group by selecting $n$ columns that are each linearly independent relevant criteria.

\paragraph{}
Proceeding column by column, we 
can see that in the first column of some matrix in this group, each element can chosen freely except that the column cannot 
be entirely zeros. Consequently, there are $q^n -1$ valid choices for the coefficients in the 
second column. In the second column any combination of coefficients can be chosen, except those 
which are a scalar multiple of the first column. Therefore, there are $g^n -q$ valid choices for 
the second column. In fact, in the $j^{th}$ column, the requirement that this column be linearly 
independent of all previous columns implies that there are $q^n$ choices, less all linear combinations 
of the previous columns (of which there are clearly $q^{j-1}$). This enables us to construct the 
following formula for $|GL_n(F_q)|$.

\begin{equation*}
	|GL_n(F_q)| = \prod_{i=1}^n(q^n-q^{i-1})
\end{equation*}

\subsection*{1.8}
$GL_n(F_{11})$ contains $SL_n(F_{11})$ where $F_{11} \cong \mathbb{Z}/11\mathbb{Z}$. Note that 
$SL_n(F_{11})$ is a normal subgroup as for $A \in GL_n{F_{11}}$ and $B \in SL_n(F_{11})$ we have 
$det(ABA^{-1}) = det(A)det(B)det(A^{-1}) = det(A)det(A)^{-1} = 1$ by properties of the determinant, 
so $ABA^{-1} \in SL_n(F_{11})$.

\paragraph{}
Then by the correspondence theorem $\exists \varPhi$ a bijection with 

\begin{equation*}
	\varPhi:\{H|SL_n(F_{11}) < H < GL_n(F_{11})\} \rightarrow \left\{\tilde H: \tilde H < \frac{GL_n(F_{11})}{SL_n(F_{11})}\right\}
\end{equation*}

This bijection preserves normality, so that if $\tilde H$ is normal then it also follows that 
$H$ is also normal. It is given in the notes that $GL_n(F_{11})/SL_n(F_{11})$ is isomorphic to $F_{11}$. Then note that there exists an isomorphism from $F_{11}$ to $\mathbb{Z}/11\mathbb{Z}$ 
and that isomorphisms preserve normality (if $ghg^{-1} \in H$ for $h \in H$ then $\phi(ghg^{-1}) = \phi(g)\phi(h)\phi(g)^{-1} \in \phi(H)$)
so all the normal subgroups of $GL_n(F_{11})$ which contain $SL_n(F_{11})$ correspond to a subgroup of 
$(\mathbb{Z}/11\mathbb{Z})^\times$.

\paragraph{}
Note that this group is equivalent to a cyclic group of order 10 (e.g. choose $6$ as the generator) 
and so all subgroups must have order 1, 2, 5, or 10. Clearly there is only group of order 1 and 
one group of order 10 as these groups are the trivial subgroup and the entire group respectively. 
The groups generated by iterating one of the other elements in this group can be discerned from the 
multiplication table below

\begin{align*}
	&2^2 = 4, \quad 2^3 = 8, \quad 2^4 \cong 5, \quad 2^5 = 10, \quad 2^6 \cong 9, \quad 2^7 \cong 7, \quad 2^8 \cong 3, \quad 2^9 \cong 6, \quad 2^{10} \cong 1 \\
	&3^2 = 9, \quad 3^3 \cong 5, \quad 3^4 \cong 4, \quad 3^5 \cong 1 \\
	&4^2 \cong 5, \quad 4^3 \cong 9, \quad 4^4 \cong 3, \quad 4^5 \cong 1 \\
	&5^2 \cong 3, \quad 5^3 \cong 4, \quad 5^4 \cong 9, \quad 5^5 \cong 1 \\
	&6^2 \cong 3, \quad 6^3 \cong 7, \quad 6^4 \cong 9, \quad 6^5 \cong 10, \quad 6^6 \cong 5, \quad 6^7 \cong 8, \quad 6^8 \cong 4, \quad 6^9 \cong 2, \quad 6^{10} \cong 1 \\
	&7^2 \cong 5, \quad 7^3 \cong 2, \quad 7^4 \cong 3, \quad 7^5 \cong 10, \quad 7^6 \cong 4, \quad 7^7 \cong 6, \quad 7^8 \cong 9, \quad 7^9 \cong 8, \quad 6^{10} \cong 1 \\
	&8^2 \cong 9, \quad 8^3 \cong 6, \quad 8^4 \cong 4, \quad 8^5 \cong 10, \quad 8^6 \cong 3, \quad 8^7 \cong 2, \quad 8^8 \cong 5, \quad 8^9 \cong 7, \quad 8^{10} \cong 1 \\
	&9^2 \cong 4, \quad 9^3 \cong 3, \quad 9^4 \cong 5, \quad 9^5 \cong 1 \\
	&10^2 \cong 1
\end{align*}

So clearly there is only one subgroup of order 2: $\{1, 10\}$ and one subgroup of order 5: $\{1,3,4,5,9\}$. 
Thus there are four subgroups (which are all normal as this group is Abelian). Therefore 
$F_{11}^\times$ has four normal subgroups and consequently there are four subgroups of $GL_n(F_{11})$ 
which contain $SL_n(F_{11})$ and all of them are normal.

\subsection*{1.9}
We wish to find all homomorphisms from $S_3 \rightarrow S_4$. If $\phi:S_3 \rightarrow S_4$ is 
a homomorphism then $ker(\phi) \triangleleft S_3$ so all homomorphisms must have as a kernel 
a normal subgroup of $S_3$. $S_3$ has six subgroups (two of order two and one of order 
three, the trivial group and the group itself). Each group of order two is not normal as 

\begin{equation*}
	(12)\{1,(23)\}(12) = \{1, (13)\} 
\end{equation*}

Note this holds for any subgroup of order two as our choice of elements is arbitrary (e.g. it is clear 
that one can replace $(12)$ and $(13)$ with any other pair of distinct order two elements of $S_3$ and 
obtain an identical result, see 1.6 for details). Then the 
subgroup of order 3 elements is normal as for $(12)(123)(12) = (132)$ (see 1.6 above). So it follows that each 
homomorphism is easily decomposable (by the homomorphism theorem) to composition of functions, 
one of which is an homomorphism on $S_3: g \rightarrow g/ker(\phi)$ and one of which is an isomorphism 
between $S_3$ and $im(\phi)$. Therefore we can reduce our problem to finding all unique isomorphisms 
between $S_3/H$ and $G < S_4$ where $H$ is a normal subgroup of $S_3$. The groups $S_3/H$ are as 
follows; $S_3/1 \cong S_3$, $S_3/S_3 \cong 1$ and $S_3/\{1, (123), (132)\} \cong \{1, a\}$ (where $a^2 = 1$). 

\paragraph{}
If $\phi: S_3 \rightarrow 1$ this is clearly a unique homomorphism. If $\phi : S_3 \rightarrow \{1, a\}$ 
then this homomorphism must map to a group of order 2 in $S_4$. This group will therefore contain 
an element of order 2 and the identity. There are six two-cycles in $S_4$, $(12),(13),(14),(23),(24),(34)$ 
and there are three combinations of two-cycles which have no elements in common (any combination of 
two-cycles where they share a vertex in common is equivalent to an element of order 3). So there are 
nine possible possible homomorphisms which map $S_3$ to an element a group of order two in $S_4$. 

\paragraph{}
Finally a homomorphism $\phi$ may map $S_3$ to $im(\phi) \cong S_3 < S_4$. Therefore we 
need a subgroup with three elements of order two and two elements of order three with the relation 
that if $a$ and $b$ have order two then $ab$ has order three. Clearly as well as $S_3$ there is 
also the following three groups $\{1,(14),(24),(12),(124),(142)\}$, $\{1,(24),(34),(23),(234),(243)\}$, 
$\{1,(14),(13),(14),(134),(143)\}$. 

\paragraph{}
Note that it is impossible to construct a group of isomorphic to $S_3$ that contains an element 
of order two made up of two disjoint two-cycles. This is clear as the product of three 
two cycles in $S_4$ (two of which are mutually disjoint) produces an element of order four (e.g. 
$(12)(34)(13) = (1432)$ and $(13)(34)(12) = (1234)$) and the product of two pairs of disjoint 
two cycles produces an element of order two (e.g. $(12)(34)(13)(24) = (14)(23)$). By renaming 
vertices in this permutation (e.g. by replacing 1 with 2 and visa-versa) it is straightforward 
to show this holds true for any possible way to combine elements of $S_4$ with these properties.
Consequently any subset of $S_4$ which contains an element comprised of two disjoint two-cycles 
cannot be isomorphic to $S_3$.

\paragraph{}
Therefore if $im(\phi) \cong S_3$ it must be equal to one of these these four groups. 
Since $\phi$ is a homomorphism it must map two-cycles in $S_3$ to elements of order two in $im(\phi)$. 
For each of the possible sets comprising $im(\phi)$ there are clearly six different ways to map 
the two cycles to each other (as there are 3! ways to permute three elements). Then since each 
element of order three is a product of two two-cycles the mapping is fully determined 
by the choice of which two-cycles in $S_3$ are mapped to which two-cycles in $im(\phi)$, so there 
are 24 different homomorphisms which map $S_3$ to $im(\phi) \cong S_4$.

\paragraph{}
Therefore 
there are $9 + 1 + 24 = 34$ possible homomorphisms from $S_3$ to $S_4$. 

\subsection*{1.10}
It is convenient to consider Abelian and non-Abelian groups separately.

\paragraph{Abelian Groups}
By Corollary 8.7 of the notes, we can see that if $G$ is an Abelian group of order 8, it 
must be isomorphic to a direct product of groups $H_1,\, H_2,\dots, H_k$ with $H_i \cong \mathbb{Z}/n_i\mathbb{Z}$. 
Since the order of $G$ is clearly the product of the orders of $H_1,\, H_2, \dots, H_n$ it follows 
that $\prod_{i=1}^{k}n_i = 8$. Therefore, all possible Abelian groups must be isomorphic to 
one of 

\begin{align*}
	M_1 &\cong \frac{\mathbb{Z}}{8\mathbb{Z}} \\
	M_2 &\cong \frac{\mathbb{Z}}{2\mathbb{Z}}\times\frac{\mathbb{Z}}{4\mathbb{Z}} \\
	M_3 &\cong \frac{\mathbb{Z}}{2\mathbb{Z}}\times \frac{\mathbb{Z}}{2\mathbb{Z}} \times \frac{\mathbb{Z}}{2\mathbb{Z}}
\end{align*}

It remains to show that none of these groups are not isomorphic to each other. $M_1$ contains 
elements of order 8, but for $(a_1,a_2) \in M_2$ we have $4(a_1,a_2) = (4a_1,4a_2) = 1$ as each 
element in $\mathbb{Z}/2\mathbb{Z}$ or $\mathbb{Z}/4\mathbb{Z}$ has at most order four. Therefore 
$M_2$ contains no elements of order 4 and consequently it cannot be isomorphic to $M_1$. By an 
identical argument we can see that for $(a_1,a_2,a_3) \in M_3$ we have $2(a_1,a_2,a_3) = (2a_1,2a_2,2a_3) = 1$. 
Thus $M_1$ cannot be isomorphic to $M_3$. Furthermore, since there are elements in $\mathbb{Z}/2\mathbb{Z} \times \mathbb{Z}/4\mathbb{Z}$ 
which have order four (e.g. $(1,1)$ has order four) it follows that $M_2$ and $M_3$ are not isomorphic. 
Thus we can conclude that each of these groups are not isomorphic to each other and so there are 
three Abelian groups of order 8. 

\paragraph{Non-Abelian Groups}
By Langrange's theorem, we know that even non-Abelian groups of order 8 can only contain elements of order 
2, 4 and 8. In fact, if a group of order 8 contains an element of order 8 this clearly implies that 
it is isomorphic to the cyclic group of order 8 (which is Abelian). So we can limit ourselves to 
considering non-Abelian groups containing only elements of 4 and 2. First consider the case 
where for all $a,b$ elements of this group of order two we have that $ab$ is also of order two. Then $abab = e$ so $(ab)^{-1} = ab$ but 
by definition $(ab)^{-1} = b^{-1}a^{-1} = ba$ so $ab = ba$. Therefore it follows that the group 
is in fact Abelian and that any non-Abelian group must in fact include an element of order four. 

\paragraph{}
Let $G$ be a non-Abelian group of order 8 with $a$ an element of order four. Suppose 
that then there exists some element $b$ of order two, which is not in $<a>$. This implies that the 
eight elements of this group can be written as \{$1$, $a$, $a^2$, $a^3$, $ab$, $ab^2$, $ab^3$, $b$\}. Since 
the group has eight elements in it, it follows that $ba$, $ba^2$, $ba^3$ 
must each be mapped to some element inside this set. If $ba^n = a^m$ for some $n,m \in \{1,2,3\}$ 
then $b = a^{n-m}$ by cancellation, which is a contradiction of our initial hypothesis that $b$ 
is not in $<a>$. Therefore $ba^n = a^mb$ for each $n,m \in \{1,2,3\}$.


\paragraph{}
If $ba = ab$ then $a^2b = aba = ba^2$ and $a^3b = ba^3$ so the whole 
group is Abelian. If $ba = a^2b$ then $a = ba^2b$ and so $a^2 = ba^2bba^2b = 1$ which is a 
contradiction. Therefore $ba = a^3b$. If $ba^2 = ab$ then $a^2 = bab$ so $a^4 = ba^2b = 1$ 
so $a^2 = 1$ which is a contradiction. This implies that $ba^2 = ab^2$ (as trivially $ba^2 = a^3b$ 
implies that $ba = ba^2$ which is an obvious contradiction). Finally this implies that $ba^3 = ab$ 
and so this group can be expressed as $< a,b \quad | \quad  a^4 = b^2 = 1, \, ab = ba^{-1} >$ 
(note this is isomorphic to the Dihedral group of order 4).

\paragraph{}
Now we will consider the case where all elements in $G$ that are not in the subgroup 
$<a>$ have order four. Then for $b \neq a^n$ for some $n \in [0,4]$ it follows that $b$ has order 
four and that $b^2$ has order two. This implies that $b^2 = a^2$ by our assumption. This suggests 
that the group contains the following elements $1$, $a$, $a^2$, $a^3$, $b$, $ab$, $a^2b$, $a^3b$. 
Since this group has order 8 it must also be true that $ba$, $ba^2$ and $ba^3$ are equivalent 
to elements inside this group. Clearly $a^2b = b^3 = ba^2$. Then if $ba = ab$ and $ba^3 = a^3b$ then 
this group is Abelian, but by assumption it was not. Therefore $ba = a^3b$ and $ab = ba^3$. 
This forms a closed group, which can be expressed as $< a,b \quad | \quad a^4 = b^4 = 1,\, a^2 = b^2,\, ab = ba^3 >$ 
(note that $ab = ba^3 \implies a^3b = ba$ as $a^3b = a^2ba^3 = b^3a^3 = ba^5 = ba$ as required). 
This is isomorphic to the Quarternion group. 

\paragraph{}
There are therefore three Abelian group and two Non-Abelian groups of order 8, for a total of 
five groups. Groupprops suggests there are 267 groups of order 64 \cite{Groupprops}.

\printbibliography
\end{document}