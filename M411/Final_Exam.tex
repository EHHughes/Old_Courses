\documentclass{article}

\title{	
	\normalfont\normalsize 
	\rule{\linewidth}{0.5pt}\\ % Thin top horizontal rule
	\vspace{14pt} % Whitespace
	{\LARGE MATH411 Final Exam\\ % The assignment title
    \large \textit{} \\}
	\vspace{6pt} % Whitespace
	\rule{\linewidth}{1pt}\\ % Thick bottom horizontal rule
}

\author{Elliott Hughes}
\date{\normalsize\today}
\usepackage{tikz}
\usetikzlibrary{arrows,automata}
\usetikzlibrary{positioning}
\usetikzlibrary{arrows.meta,positioning}
\usepackage{mdframed}
\usepackage{amsmath}
\usepackage{amssymb}
\usepackage{graphicx}
\graphicspath{ {./Images/} }
\usepackage{commath}
\usepackage{textcomp}
\usepackage{gensymb}
\usepackage{float}
\usepackage{hyperref}
\usepackage[margin=1in]{geometry}
\usepackage{caption}
\usepackage{subcaption}
\usepackage{sectsty}
\usepackage{titlesec}
\def\acts{\rotatebox[origin=c]{-60}{$\circlearrowright$}}

\begin{document}

\maketitle

\section*{Q1}
\subsection*{(a)}
It is straightforward to show the conjugacy classes of the generators of $D_6$. First 
$\rho^q\rho^n\rho^{-q} = \rho^n$ for $n,q \in \mathbb{Z}$ and $\sigma\rho^n\sigma = \rho^{-n}$. 
Finally $\sigma\rho^q\rho^n\rho^{-q}\sigma = \rho^{-n}$. Since $\rho^q\sigma = \rho^{-q}\sigma$ 
this is sufficient to demonstrate that $\rho^n$ is conjugate to $\rho^{-n}$ and no other elements. 

\paragraph{}
Furthermore $\sigma\rho^n\sigma\sigma = \rho^{-n}\sigma$, $\rho^q\rho^n\sigma\rho^{-q} = \rho^{2q+n}\sigma$ 
and $\sigma\rho^q\rho^n\sigma\rho^{-q}\sigma = \sigma\rho^{2q+n}\sigma\sigma = \rho^{-2q-n}\sigma$. By an 
identical argument as above this limits the possible conjugacy classes to the following set 

\begin{align}
    &\{\rho,\rho^5\} \\
    &\{\rho^2,\rho^4\} \\
    &\{\rho^3\} \\
    &\{\sigma,\rho^2\sigma,\rho^4\sigma\} \\
    &\{\rho\sigma,\rho^3\sigma,\rho^5\sigma\} \\
    &\{e\}
\end{align}

So the center is simply the set of conjugacy classes of size one, or $Z(D_6) = \{e,\rho^3\}$.

\subsection*{(b)}
Consider the set $S = \{e,\sigma\rho^2,\sigma\rho^4,\sigma,\rho^2,\rho^4\}$. The Cayley table of 
this set is 

\begin{center}
    \begin{tabular}{ c | c c c c c c } 
    & $e$ & $\sigma$ & $\sigma\rho^2$ & $\sigma\rho^4$ & $\rho^2$ & $\rho^4$ \\
    \hline
    $e$ & $e$ & $\sigma$ & $\sigma\rho^2$ & $\sigma\rho^4$ & $\rho^2$ & $\rho^4$ \\
    $\sigma$ & $\sigma$ & $e$ & $\rho^2$ & $\rho^4$ & $\sigma\rho^2$ & $\sigma\rho^4$ \\
    $\sigma\rho^2$ & $\sigma\rho^2$ & $\rho^4$ & $e$ & $\rho^2$ & $\sigma\rho^4$ & $\sigma$ \\
    $\sigma\rho^4$ & $\sigma\rho^4$ & $\rho^2$ & $\rho^4$ & $e$ & $\sigma$ & $\sigma\rho^2$ \\
    $\rho^2$ & $\rho^2$ & $\sigma\rho^4$ & $\sigma$ & $\sigma\rho^2$ & $\rho^4$ & $e$ \\
    $\rho^4$ & $\rho^4$ & $\sigma\rho^2$ & $\sigma\rho^4$ & $\sigma$ & $e$ & $\rho^2$ 
    \end{tabular}
\end{center}

Therefore this set is a subgroup and it is isomorphic to $S_3$. Since $S_3$ has size six it has index 
two in $D_6$ and is thus normal. Furthermore $Z(D_6)$ is trivially normal in $D_6$, these sets 
have trivial intersection and $Z(D_6)S = D_6$ so $D_6 \cong Z(D_6) \times S \cong \mathbb{Z}_2 \times S_3$.

\subsection*{(c)}
Clearly any size six subgroup will have to contain a size three subgroup and this size three subgroup must 
be contained in $S_3$ so there is only one such subgroup in $D_6$ ($H_3 \cong A_3 < S_3$). Since this 
is the only size three subgroup the Sylow theorems imply that it is normal. It follows from the correspondence 
theorem that the number of index two subgroups (which must contain $H_3$) must be equal to the 
number of index two subgroups in $D_6/H_3$. It is straightforward to find $H_3 = \{e,\rho^2,\rho^4\}$ 
and so enumerate the elements of $D_6/H_3$

\begin{equation*}
    D_6/H_3 = \{H_3,\rho H_3,\sigma H_3, \rho \sigma H_3\}
\end{equation*}

Since three of these elements have order two, there must be three index two subgroups in $D_6$. So there 
are three order six subgroups of $D_6$. In particular there exists $H_{\rho} = \{e,\rho,\rho^2,\rho^3,\rho^4,\rho^5\}$ 
the subgroup of rotations, $H_{\sigma} =  \{e,\sigma,\rho^2,\rho^4,\sigma\rho^2,\sigma\rho^4\}$ 
and $H_{\rho\sigma} = \{e,\rho\sigma,\rho^2,\rho^4,\rho^3\sigma,\rho^5\sigma\}$.

\section*{Q2}
\subsection*{(a)}
It is hopefully obvious that $[\mathbb{Q}(\sqrt{2},\sqrt{-3}):\mathbb{Q}] = 4$. Since the cube 
root of unity is an element of $\mathbb{Q}(\sqrt{2},\sqrt{-3})$ the polynomial $x^3-5$ splits over 
$\mathbb{Q}(\sqrt{2},\sqrt{-3},\sqrt[3]{5})$. Straightforward enumeration shows that the maximum possible basis size 
is  
$[\mathbb{Q}(\sqrt{2},\sqrt{-3},\sqrt[3]{-5}):\mathbb{Q}]\leq 12$ and since both $[\mathbb{Q}(\sqrt{2},\sqrt{-3}):\mathbb{Q}] = 4$ 
and $[\mathbb{Q}(\sqrt[3]{5}):\mathbb{Q}] = 3$ must divide $[\mathbb{Q}(\sqrt{2},\sqrt{-3},\sqrt[3]{-5}):\mathbb{Q}]$ 
it follows that $[\mathbb{Q}(\sqrt{2},\sqrt{-3},\sqrt[3]{-5}):\mathbb{Q}] = 12$. This field 
is the splitting field of $f(x) = (x^2-2)(x^2+3)(x^3-5)$ which is separable and so $K$ is Galois.

\paragraph{}
The existence of a size 6 subgroup is trivial as $\mathbb{Q}(\sqrt{2})$ is a subfield of $K$ 
and is a quadratic extension of $\mathbb{Q}$. Therefore there exists $H_6$ fixing $\mathbb{Q}(\sqrt{2})$ 
which must be normal. Furthermore since $\mathbb{Q}(\sqrt{-3},\sqrt[3]{5})$ is the splitting field of 
the separable polynomial $f(x) = (x^2+3)(x^3-5)$ it is Galois and since $[\mathbb{Q}(\sqrt{-3},\sqrt[3]{5}):\mathbb{Q}] = 6$ 
there exists some size two subgroup which is normal in $H_2 \triangleleft \text{Aut}(K|\mathbb{Q})$.

\paragraph{}
Since an automorphism that fixes both the elements in $\mathbb{Q}(\sqrt{2})$ and elements in $\mathbb{Q}(\sqrt{-3},\sqrt[3]{5})$ 
must fix all elements in $K$ it must be true that $H_2 \cap H_6 = e$. Furthermore $H_2H_6 = \text{Aut}(K|\mathbb{Q})$ 
it follows that $H_2 \times H_6 \cong \text{Aut}(K|\mathbb{Q})$. It remains to show that $H_6$ is 
isomorphic to $S_3$. Since $H_6$ fixes $\mathbb{Q}(\sqrt{2})$ it must permute $\sqrt{-3}$, $-\sqrt{-3}$ 
and $\omega\sqrt[3]{5}$, $\omega^2\sqrt[3]{5}$, $\sqrt[3]{5}$. Complex conjugation clearly 
permutes these roots and fixes $\sqrt{2}$, $-\sqrt{2}$ so it must be an element of $H_6$. Let 
$\tau$ denote this automorphism with $H' = \{e,\tau\}$ and let $\sigma$ be an element of size three in $H_6$. Then for $a$ a not fixed by $H_6$, $\sigma\tau\sigma^{-1}(a) \neq \bar{a}$ 
for all $a$ unless $\sigma^{-1}(a) = 5^{1/3}$. So $\sigma H' \sigma^{-1} \neq H'$ and there 
must be more than one size two subgroup of $H_6$. The only group with these properties is $S_3$ so $H_6 \cong S_3$. Therefore $\text{Aut}(K|\mathbb{Q}) \cong \mathbb{Z}_2 \times S_3$ 
as required.

\subsection*{(b)}
Clearly $[\mathbb{Q}(\sqrt{2},i):\mathbb{Q}] = 4$. From an identical argument to that above 
it follows that $[\mathbb{Q}(\sqrt[5]{7}):\mathbb{Q}]\mid[E:\mathbb{Q}]$ and $[E:\mathbb{Q}] \leq 28$ 
so $[E:\mathbb{Q}] = 28$. But $E|\mathbb{Q}$ is normal only if $x^7-5$ splits in $E$. This holds 
only when the seventh primitive root of unity (denoted here $\omega$) is in $E$. This would 
imply that $\mathbb{Q}(\omega) \subseteq E$. But $[\mathbb{Q}(\omega):\mathbb{Q}] = 6$ and 
$6 \nmid 28$ so $\omega \notin E$. Therefore $E$ is not normal.

\section*{Q3}
Clearly the roots of the polynomial are given by 

\begin{equation*}
    x = \omega(2\pm \sqrt{3})^\frac{1}{3} \qquad \text{where} \quad \omega^3=1,\quad \omega \neq 1
\end{equation*}

Let $\alpha = (2 + \sqrt{3})^{\frac{1}{3}}$ and $\beta = (2-\sqrt{3})^{\frac{1}{3}}$. The splitting 
field of this polynomial over $\mathbb{Q}$ is then $E = \mathbb{Q}(\omega,\sqrt{3},\alpha,\beta)$. It is 
also useful to note that $\alpha = -1/\beta$. Clearly $[\mathbb{Q}(\omega,\sqrt{3}):\mathbb{Q}] = 4$ 
and since $\alpha$ is cube root the minimal polynomial of $\alpha$ must have a degree divisible by three so it follows that $4,3|[\mathbb{Q}(\omega,\sqrt{3},\alpha):\mathbb{Q}]$ 
and $[\mathbb{Q}(\omega,\sqrt{3},\alpha):\mathbb{Q}] \leq 12$ so $[E:\mathbb{Q}] = 12$ as $\beta \in \mathbb{Q}(\omega,\sqrt{3},\alpha)$. 

\paragraph{}
The automorphism group must then have size 12. Since $\mathbb{Q}(\omega,\sqrt{3})$ has degree 
four and is Galois there exists a normal subgroup of size three, $H_3 \cong A_3$. This implies that the 
number of Sylow three subgroups must be one and so all subgroups of size six must include this 
subgroup. The correspondence theorem 
then implies that the structure of the subgroup lattice above $H_3$ is equivalent to the structure 
of the subgroup lattice of $G/H_3$. Since $G/H_3$ has order four it must be congruent to either 
$\mathbb{Z}_4$ or $\mathbb{Z}_2 \times \mathbb{Z}_2$, which implies there are either 
one or three extensions of degree six (as a subgroup with index 2 in $G$ must match a subgroup of 
index 2 in $G/H_3$). Since there are at least two subgroups of index 2 in $G$ (as $\mathbb{Q}(\sqrt{3})$ 
and $\mathbb{Q}(\omega)$ are both quadratic extensions of $\mathbb{Q}$ and proper subfields of $E$) 
it follows that there must be exactly three such extensions. 

\paragraph{}
Consider $\sigma \in \text{Aut}(E|\mathbb{Q})$. Note furthermore that since $\sigma$ is an 
automorphism, the action of $\sigma$ on $\beta$ is determined entirely by its action on 
$\alpha$. Since $\sigma$ must permute the roots of 
$f(x) = x^6 - 4x^3 + 1$ it follows that either $\sigma(\alpha) = \omega^n\alpha$ or $\sigma(\alpha) = \omega^n\beta$ and 
that $\sigma(\beta) = -\sigma$. 
Consider then the automorphism $\tau$ taking $\alpha$ to $\beta$ (that this is an automorphism 
is straightforward to verify) it follows that for $\gamma$ a root $\sigma\tau\sigma(\gamma) = -\gamma^{-1} = \tau(\gamma)$ 
and so the set $H_2 = \{e,\tau\}$ is a size two normal subgroup of $\text{Aut}(E|F)$.

\paragraph{}
Consider furthermore the subfield $\mathbb{Q}(\sqrt{3})$. Since $\tau(2 + \sqrt{3}) = (\alpha^3) = \beta^3 = 2 - \sqrt{3}$ 
it follows that $\tau$ does not fix this subfield, so the size six subgroup $H_{\sqrt{3}}$ fixing 
this field (which has index 
two and is thus normal) and $H_2$ have trivial intersection. Therefore since together they 
compose the whole set it follows that $\text{Aut}(E|F) \cong H_2 \times H_{\sqrt{3}}$.

\paragraph{}
It is also useful to note that $f$ is irreducible (as while $(\alpha\beta)^n \in \mathbb{Q}$ for all 
$n$, $\alpha^n+\beta^n \in \mathbb{Q}$ only for $n=3$). Therefore, using the result from problem 4.2 
of the final assignment, $G$ cannot be Abelian. It thus follows that $H_{\sqrt{3}}$ must be a 
non-Abelian group of order 6, of which $S_3$ is the only possible choice. Thus $\text{Aut}(E|F) \cong \mathbb{Z}_2 \times S_3 \cong D_6$.

\section*{Q4}
Since $E|F$ is Galois with $\text{Aut}(E|F) \cong S_3$ the characterisation of Galois 
extensions implies that there exists some $g(x)$ seperable which splits over $E|F$. Let 
$g(x) = h_1(x)h_2(x)\dots h_n(x)$ where each $h_i(x) \in F[x]$ is an irreducible factor. It 
is clear that there exists $g^*(x)$ a separable polynomial with no roots in $F$ (as any linear $h_i$ 
can be eliminated from $g(x)$ and it will still split only over $E|F$). Then if $g^*$ is not 
irreducible there exists some $h_1(x),h_2(x),\dots h_n(x) \in F[x]$ which are irreducible and seperable 
polynomials. If any $h_i$ split only in $E|F$ then we have obtained the desired result.

\paragraph{}
If however, all $h_1,h_2,\dots,h_n$ split in proper subfields of $E|F$ then there must exist $h_k$, $h_j$ 
which split only in distinct subfields (as otherwise the splitting field of $g(x)$ would be 
the proper subfield rather than $E|F$). Then since $h_k$ and $h_j$ are separable the splitting fields 
of these two polynomials are themselves Galois extensions and so this implies the existence of 
two distinct normal subgroups of $S_3$ by the Galois correspondence. However, $S_3$ has only one normal 
subgroup $A_3$ and so this is a contradiction. Thus there must exist an irreducible polynomial 
that splits in $E|F$ as required.

\section*{Q5}
\subsection*{(a)}
Clearly $\sigma,\tau$ fix $K$ and for $f(x) \in K(x)$ it is clear that $\sigma(f), \tau(f) \in K(x)$. 
These functions are their own inverses and thus these inverses are well-defined everywhere 
the functions are defined, so they must be one-to-one and onto. Thus $\sigma,\tau \in \text{Aut}(K(x))$. 

\subsection*{(b)}
Note that $\sigma\tau\sigma(x) = 1 - \frac{1}{1-x}$ is also an order two automorphism. From this it 
is easy although tedious to compute the following Cayley Table:

\begin{center}
    \begin{tabular}{ c | c c c c c c } 
    & $e$ & $\sigma$ & $\tau$ & $\sigma\tau\sigma$ & $\sigma\tau$ & $\tau\sigma$ \\
    \hline
    $e$ & $e$ & $\sigma$ & $\tau$ & $\sigma\tau\sigma$ & $\sigma\tau$ & $\tau\sigma$ \\
    $\sigma$ & $\sigma$ & $e$ & $\sigma\tau$ & $\tau\sigma$ & $\tau$ & $\sigma\tau\sigma$ \\
    $\tau$ & $\tau$ & $\tau\sigma$ & $e$ & $\sigma\tau$ & $\sigma\tau\sigma$ & $\sigma$ \\
    $\sigma\tau\sigma$ & $\sigma\tau\sigma$ & $\sigma\tau$ & $\tau\sigma$ & $e$ & $\sigma$ & $\tau$ \\
    $\sigma\tau$ & $\sigma\tau$ & $\sigma\tau\sigma$ & $\sigma$ & $\tau$ & $\tau\sigma$ & $e$ \\
    $\tau\sigma$ & $\tau\sigma$ & $\tau$ & $\sigma\tau\sigma$ & $\sigma$ & $e$ & $\sigma\tau$ 
    \end{tabular}
\end{center}

So this group $G$ is isomorphic to $S_3$.

\subsection*{(c)}
It is possible although tedious to show $\sigma(u) = u$ and $\tau(u) = \tau$, so 
$K(u) \subset E^{G}$. Consider the polynomial $f(y) \in K(u)[y]$,

\begin{equation*}
    f(y) = (y^2+u)(y^4-2y^3+y^2)-y^5-y^4+5y^3-3y+1
\end{equation*}

When $y= x$ this polynomial becomes the zero polynomial (in $x$), so $x$ is a root of $f(y)$. 
Since any polynomial in $x$ that has $x$ indeterminate as a root must be the zero polynomial, 
any polynomial $f(y) \in K(u)[y]$ that has $y =x$ as a solution and is not the zero polynomial in 
$K(u)[y]$ must include a power of $u$ as a coefficient. Since we furthermore require that $f(x)$ 
be a polynomial this implies that the minimum degree must be the largest degree in the numerator of 
$u$ (as since the numerator also includes a constant term one cannot divide by a non-constant 
rational function and obtain a polynomial in $x$). Therefore $[K(x):K(u)] = 6$. 

\paragraph{}
Artin's lemma then implies that $[E:E^{G}] \leq 6$ and since $[E:K(u)] = 6$ it follows that 
$[E^G:K(u)] = 1$ and so $E^G = K(u)$.


\end{document}