\documentclass{article}

\title{	
	\normalfont\normalsize 
	\rule{\linewidth}{0.5pt}\\ % Thin top horizontal rule
	\vspace{14pt} % Whitespace
	{\LARGE MATH411 Assignment 2\\ % The assignment title
    \large \textit{} \\}
	\vspace{6pt} % Whitespace
	\rule{\linewidth}{1pt}\\ % Thick bottom horizontal rule
}

\author{Elliott Hughes}
\date{\normalsize\today}
\usepackage{tikz}
\usetikzlibrary{arrows,automata}
\usetikzlibrary{positioning}
\usetikzlibrary{arrows.meta,positioning}
\usepackage{mdframed}
\usepackage{amsmath}
\usepackage{amssymb}
\usepackage{graphicx}
\graphicspath{ {./Images/} }
\usepackage{commath}
\usepackage{textcomp}
\usepackage{gensymb}
\usepackage{float}
\usepackage{hyperref}
\usepackage[margin=1in]{geometry}
\usepackage{caption}
\usepackage{subcaption}
\usepackage{sectsty}
\usepackage{titlesec}

\begin{document}

\maketitle

\subsection*{2.1}
If $M$ is a finite Abelian group with $|M| = 600$ then by the fundamental theorem of finite 
Abelian groups $M$ is isomorphic to a direct product of groups of the form $\mathbb{Z}/n_i\mathbb{Z}$, 
where $\prod_i n_i = 600$ and each $n_i$ is a prime power. Since $|M| = 600 = 2^3\cdot 3 \cdot 5^2$ it follows that $M$ can be factored into a 
direct product of an Abelian group of order $2^3$, an Abelian group of order 3 and an Abelian group of order $5^2$. The 
group of order $2^3$ (denoted $G_2$) must be isomorphic to some direct product of Abelian groups of order $2^r$, $r \in \{1,2,3\}$. 
There are clearly only three possible options 

\begin{equation*}
    G_2 = \frac{\mathbb{Z}}{2\mathbb{Z}} \times \frac{\mathbb{Z}}{2\mathbb{Z}} \times \frac{\mathbb{Z}}{2\mathbb{Z}} \quad \text{or} \quad G_2 = \frac{\mathbb{Z}}{2\mathbb{Z}} \times \frac{\mathbb{Z}}{4\mathbb{Z}} \quad \text{or} \quad G_2 = \frac{\mathbb{Z}}{8\mathbb{Z}} 
\end{equation*}

These are the only possible groups of order eight which are Abelian by the fundamental theorem of 
finite Abelian groups. Clearly the only possible group of order three is just 

\begin{equation*}
    G_3 = \frac{\mathbb{Z}}{3\mathbb{Z}}
\end{equation*}

Finally there are two possible groups of order $5^2$. 

\begin{equation*}
    G_5 = \frac{\mathbb{Z}}{5\mathbb{Z}} \times \frac{\mathbb{Z}}{5\mathbb{Z}} \quad \text{or} \quad G_5 = \frac{\mathbb{Z}}{25\mathbb{Z}} 
\end{equation*}

Consequently there are $3(2) = 6$ unique Abelian groups of order 600.

\subsection*{2.2}
Consider $\phi,\psi \in \text{Aut}(G)$. Then for $g,h \in G$ we have $\phi \circ \psi(gh) = \phi(\psi(g)\psi(h))$. 
Since $\psi(g),\psi(h) \in G$ it is clear that $\phi\psi(gh) = \phi\psi(g)\phi\psi(h) \in G$. Therefore the set $\text{Aut}(G)$ is closed under the operation $\phi \times \psi \rightarrow \phi \circ \psi$. 
Let $\psi_e$ be the function taking $g \rightarrow g$ for all $g \in G$. This is an automorphism as 
clearly for $g,h \in G$ we have $\psi_e(gh) = \psi_e(g)\psi_e(h)$ and $\phi_e$ is trivially bijective. Then for $\phi \in \text{Aut}(G)$ and 
$g \in G$ we have $\phi \circ \psi_e(g) = \phi(\psi_e(g)) = \phi(g)$ so $\phi \circ \psi_e = \phi$. Thus 
$\psi_e$ is a valid identity element for the candidate group $\text{Aut}(G)$.
Since $\phi$ is an isomorphism from $G$ to $G$, the inverse function $\phi^{-1}$ is also an isomorphism 
from $G$ to $G$. Furthermore for $g \in G$, $\phi\circ\phi^{-1}(g) = \phi(\phi^{-1}(g)) = g$ so 
$\phi\circ\phi^{-1} = \psi_e$. Finally by the properties of functions for $\phi_1,\phi_2,\phi_3 \in \text{Aut}(G)$, 
$(\phi_1 \circ \phi_2) \circ \phi_3 = \phi_1 \circ (\phi_2 \circ \phi_3)$ and so $\text{Aut}(G)$ is a valid 
group.

\paragraph{}
Then for some $g \in G$ let $\phi_g: G \rightarrow G$, such that for $h \in G$, $\phi_g(h) = ghg^{-1}$. 
If we take $h_1,h_2 \in G$ we have $\phi(h_1h_2) = gh_1h_2g^{-1} = gh_1g^{-1}gh_2g^{-1} = \phi_g(h_1)\phi_g(h_2)$ 
so $\phi_g$ is a homomorphism from $G$ to $G$. Clearly $\phi_g$ is injective as for $h_1,h_2 \in G$ 
then if $gh_1g^{-1} = gh_2g^{-1}$ it follows that $h_1 = h_2$. It is also surjective as if 
$h \in G$ then there exists $g^{-1}hg \in G$ such that $\phi(g^{-1}hg) = h$. So $\phi_g$ is an 
automorphism.

\paragraph{}
We may now define the function $\Phi:G\rightarrow \text{Aut}(G)$, $\Phi(g) = \phi_g$. It remains to show that this is 
a homomorphism. Consider $g_1,g_2 \in G$. Then $\Phi(g_1g_2) = \phi_{g_1g_2}$ and for $h \in G$ 
$\phi_{g_1g_2}(h) = g_1g_2h(g_1g_2)^{-1} = \phi_{g_1} \circ \phi_{g_2}(h)$. Since this holds for 
all $h \in G$, it follows that $\phi_{g_1g_2} = \phi_{g_1} \circ \phi_{g_2}$ and thus $\Phi$ is a 
homomorphism. Clearly the kernel of $\Phi$ includes the group of all elements which commute with any 
element in $G$ (as if $g \in Z(G)$ then for any $h \in G$ we have $ghg^{-1} = gg^{-1}h = h$) so 
$\text{ker}(\Psi) = Z(G)$. Furthermore if for $ghg^{-1} = h$ for all $h \in G$ then clearly $gh = hg$ and 
so $g \in Z(G)$ so $\text{ker}(\Phi) = Z(G)$.

\paragraph{}
This is not necessarily a surjective function, however. Let $G$ be an Abelian group with elements 
which have order higher than two. Consider $\phi:G\rightarrow G$, $\phi(g) = g^{-1}$ 
for some $g \in G$. This is clearly injective by the uniqueness of inverses and onto as for $g \in G$ 
there exists $(g^{-1})^{-1} = g$. Finally for $g,h \in G$ so that $\phi(gh) = (gh)^{-1} = h^{-1}g^{-1} = g^{-1}h^{-1} = \phi(g)\phi(h)$ 
so this an automorphism. Then if we consider $\Phi$ from this group $G$ to $\text{Aut}(G)$ it 
is clear that $\Psi$ maps only to the identity (as $Z(G) = G$) and since $G$ has elements which 
have a higher order than two it follows that $\phi$ is not in the image of $\Phi$. Thus $\Phi$ is 
not surjective.

\subsection*{2.3}
Note $A_4$ is a subgroup and is of index two, so it is normal (see 2.4). Furthermore $A_4$ is clearly of maximal 
size as there are no divisors for 24 which are larger than 12. So $A_4$ is a valid first term in a 
composition series. Then, consider $V_4$ the subgroup of $A_4$ comprised of the identity and three 
pairs of 2-cycles. This subgroup contains all the elements of order 2 in $v \in V_4$, so for 
$g \in V_4$ we have $gvg^{-1}$ is also an element of order 2, so $V_4$ is normal. Any larger 
subgroup containing $V_4$ would have to be of order six or 12. It is trivial to see that if one introduces multiplies a 
3 cycle by the elements of $V_4$ more than two new elements are created, so this series $S_4 \triangleright A_4 \triangleright V_4$ 
is part of a valid composition series for $S_4$. Clearly since any cyclic subgroup of $V_4$ generated by a 
non-identity element $a$ is both normal and maximal (any larger subgroup is the group itself) it follows that 
the final composition series of $A_4$ is 

\begin{equation*}
    S_4 \triangleright A_4 \triangleright V_4 \triangleright \langle a \rangle \triangleright {1}
\end{equation*}

Is a valid composition series for $S_4$. The consequent J\"ordan-Holder factors are 

\begin{equation*}
    S_4/A_4 \cong \mathbb{Z}_2 \qquad A_4/V_4 \cong \mathbb{Z}_3 \qquad V_4/\langle a\rangle \cong \mathbb{Z}_2 \qquad \langle a \rangle/{1} \cong \mathbb{Z}_2
\end{equation*}

\subsection*{2.4}
Let $G$ be a group. If $G$ is a $p$-group then the desired result holds by Corollary 11.6. Otherwise 
$|G| = p^rm$ where $p$ is the smallest prime dividing the order of $G$, $m$ is co-prime to $p$ and $r \geq 1$. 
Then let $H < G$ such that $|H| = p^{r-1}m$. Denote the set of cosets of $H$ (of which there are $p$) as 

\begin{equation*}
    S = \{H,\,g_1H,\dots g_{p-1}H\}
\end{equation*}

Then we can define the permutation group of this set as $\text{Sym}(S)$. If one multiplies each element of 
$S$ by some element of $G$ then each coset is either fixed or sent to a new coset and so this 
defines a permutation $T_g$. Consider the function $\phi:g\rightarrow T_g$. Then for $g,g' \in G$ 
we have $\phi(gg') = T_{gg'}$ and so for some $s \in S$, $T_{gg'} = gg'(s) = g(g'(s)) = T_g(T_{g'}(s))$ 
and so $\phi(gg') = \phi(g)\phi(g')$. Therefore $\phi$ defines a homomorphism between these two 
groups. 

\paragraph{}
Since $\text{Im}(\phi)$ is a subgroup of $\text{Sym}(S)$ the order of $\text{Im}(\phi)$ must divide 
$p!$. By the first isomorphism theorem this implies that $G/\text{ker}(\phi)$ must also divide 
$p!$. Clearly the order of $G/\text{ker}(\phi)$ cannot have prime factors larger than $p$. Consequently 
we have that $[G:\text{ker}(\phi)]$ must be some power of $p$ as $p$ is the smallest prime that 
divides the order of $G$. If $[G:\text{ker}(\phi)] = p^k$, then $k = 0,1$ as $p$ only appears once 
in $p!$.

\paragraph{}
If $[G:\text{ker}(\phi)] = 0$ then the kernel of $\phi$ is the whole group. This would imply that 
for all $g \in G$, $T_g$ is simply the identity permutation. However if one chooses $g \notin H$ 
then clearly $gH$ is not the same as $H$ so the kernel of $\phi$ is not the whole group. Thus 
we have that $[G:\text{ker}(\phi)] = p$. It should be clear from the argument above that if $g \notin H$ 
then $g \notin \text{ker}(\phi)$ and consequently that $\text{ker}(\phi) \subseteq H$. However 
since both subgroups have the same index in the group we have that $\text{ker}(\phi) = H$ and 
thus $H$ is normal in $G$.

\subsection*{2.5}
Let $G$ be a finite group with $|G| = 10 = 2\cdot5$. Consider $G \circlearrowright G$ by conjugation. 
The orbit-stabilizer theorem requires that the orbit of an element under conjugation (its conjugacy 
class) divides the order of the group. Therefore we can eliminate the third tuple as a possible candidate. Furthermore, since the 
set of all elements that commute with all elements $Z(G)$ (that is, those where $\text{Stab}(x) = G$) is a subgroup its order must divide the 
order of the group. Thus we can eliminate the first tuple which has three elements in $Z(G)$.

\paragraph{}
Consider the fourth tuple. This contains a non-identity element which commutes with all other 
elements. However, if $a,b \in G$ and $|a| = 2$ and $|b| =5$ then if these commute the product 
$ab$ must have order 10 (as $(ab)^5 = a^5b^5 = a$ and $(ab)^2 = a^2b^2 = b^2$). However then 
$ab$ must commute with all other elements as all other elements can be expressed as powers of $ab$. 
This leads to a contradiction and so we can eliminate the fourth tuple.

\paragraph{}
This leaves us with the second tuple. Consider the group $D_5$. If $s$ is a reflection from this 
group and $r$ is a rotation then the identity $srs = srs^{-1} = r^{-1}$ demonstrates that $r$ and 
$r^{-1}$ are conjugate. Furthermore since the stabilizer of $r$ is the group of rotations all rotations 
have only two elements in their orbit and thus belong to a conjugacy class of order two. Furthermore, 
since $rsr^{-1} = rrs = r^2s$ it is clear that rotations do not fix reflections under conjugations 
and since for $s'$ a different reflection $s'ss' \neq s$ the stabilizer of $s$ must be of size 
two. Consequently the orbit must be of size five and so all reflections must be in the same 
conjugacy class (as the orbits of elements in $G$ must partition $G$). Therefore the orders of 
the conjugacy classes of $D_5$ match the second tuple and so this is a valid choice.

\subsection*{2.6}
Let $\sigma$ be an r-cycle in $S_n$ and let $\tau \in S_n$. Then clearly $\tau\sigma\tau^{-1} \in S_n$. 
Suppose $\sigma = (i_1i_2i_3\dots i_r)$. Then for $\tau(i_k)$, $k \in \{1,2,\dots,r-1\}$ we have 

\begin{equation*}
    \tau\sigma\tau^{-1}(\tau(i_k)) = \tau\sigma(i_k) = \tau(i_{k+1})
\end{equation*}

And for $\tau(i_r)$, $\tau\sigma\tau^{-1}(\tau(i_r)) = \tau(i_1)$ by direct computation. Suppose then 
that $j \notin \{\tau(i_1),\tau(i_2),\dots,\tau(i_r)\}$. Then 

\begin{equation*}
    \tau\sigma\tau^{-1}(j) = \tau\tau^{-1}(j) = j
\end{equation*}

Thus $\tau\sigma\tau^{-1} = (\tau(i_1)\tau(i_2)\dots\tau(i_r))$ as required.

\subsection*{2.7}
Suppose $P$ and $Q$ are normal Sylow-P-Subgroups for different primes. Since all elements of $P$ 
have orders which do not divide the order of $Q$ and visa-versa they must have only trivial intersection. 
Thus we can apply lemma 7.2 and so their elements must commute.

\subsection*{2.8}
Clearly $351 = 3^2\cdot 13$ so any group of order 351 will contain a Sylow-13-Group. In fact there 
will be $s_{13}$ of these, where $s_{13} \cong 1$ mod 13 and $s_{13} | 9$. Clearly then $s_{13} = 1$ so this 
subgroup is normal. Therefore this group is not simple.

\subsection*{2.9}
Note $72 = 3^2\cdot2^3$. There exists $s_3$ groups of 
order 9, where $s_3 \cong 1$ mod 3 and $s_3 | 8$. Therefore $s_3$ is either 1 or 4. 

\paragraph{}
Suppose $s_3 = 4$. Then for some Sylow-3-Group $H$ we can apply the Orbit-Stabilizer theorem to 
note this implies that there exists some group $\text{Stab}(H) = K < G$ such that $[G:K] = 4$. Consider the 
set of permutations of the cosets in $K$ (as in 2.4). This group clearly has $4! = 24$ elements. 
Then one can define a homomorphism $\phi$ between $G$ and $K$ (again as in 2.4). Since $G$ has 
72 elements and $\text{Sym}(K)$ has 24 elements it follows that $[G:\text{ker}(\phi)] \leq 3$ (since $|G/\text{ker}(\phi)|$ 
has to be smaller than $|\text{Sym}(K)|$ by the first isomorphism theorem). Therefore there must exist some 
non-trivial kernel of this homomorphism and consequently if $s_3 = 4$ then $G$ is not normal. 

\paragraph{}
If $s_3 =1$ then clearly there only exists one Sylow-3-Group and this group must be normal. Thus 
there are no simple groups of order 72.

\subsection*{2.10}
Since the prime-factorization of $20449$ is $11^2\cdot13^2$, then for $G$ such that $|G| = 20449$ $G$ must contain $s_{11}$ Sylow-11-Groups 
and it must also contain $s_{13}$ Sylow-13-Groups. Furthermore $s_{11}$ must be that $s_{11} \cong 1$ mod 11 
and $s_{11} | 169$. Since the only divisors of 169 are 1, 13 and 169, it follows that $s_{11} = 1$. 
Also $s_{13}$ must be such that $s_{13} \cong 1$ mod 13 and $s_{13} | 121$. By an identical argument as 
above, it follows that $s_{13} = 1$.

\paragraph{}
Let $H_{11}$ denote the subgroup of $G$ with order 121 and $H_{13}$ denote the subgroup of $G$ with 
order 169. Since both are the only elements of their conjugacy classes, it follows that $H_{11}$ and 
$H_{13}$ are normal in $G$. Furthermore, since they have trivial intersection and $|G| = |H_{11}||H_{13}|$ 
it follows that $G \cong H_{11} \times H_{13}$.

\paragraph{}
Since this holds for any group $G$ with order 20449, it remains to show to classify the groups 
of order 121 and 169. Since both these groups are of order $p^2$ for some prime $p$, by corollary 
11.5 these groups are Abelian. Therefore by the fundamental theorem of Abelian groups, it must be 
true that $H_{11}$ and $H_{13}$ are isomorphic to a direct product of groups of integers mod $p$ 
where $p$ is the relevant prime. Therefore $H_{11}$ must be one of two possible groups

\begin{equation*}
    H_{11} \cong \frac{\mathbb{Z}}{11\mathbb{Z}} \times \frac{\mathbb{Z}}{11\mathbb{Z}} \quad \text{or} \quad H_{11} \cong \frac{\mathbb{Z}}{121\mathbb{Z}}
\end{equation*}

and $H_{13}$ must be one of

\begin{equation*}
    H_{13} \cong \frac{\mathbb{Z}}{13\mathbb{Z}} \times \frac{\mathbb{Z}}{13\mathbb{Z}} \quad \text{or} \quad H_{13} \cong \frac{\mathbb{Z}}{169\mathbb{Z}}
\end{equation*}

Thus there are only four possible groups of order 20449:

\begin{align*}
    G &\cong \frac{\mathbb{Z}}{11\mathbb{Z}} \times \frac{\mathbb{Z}}{11\mathbb{Z}} \times \frac{\mathbb{Z}}{13\mathbb{Z}} \times \frac{\mathbb{Z}}{13\mathbb{Z}} \\
    G &\cong \frac{\mathbb{Z}}{121\mathbb{Z}} \times \frac{\mathbb{Z}}{13\mathbb{Z}} \times \frac{\mathbb{Z}}{13\mathbb{Z}} \\
    G &\cong \frac{\mathbb{Z}}{11\mathbb{Z}} \times \frac{\mathbb{Z}}{11\mathbb{Z}} \times \frac{\mathbb{Z}}{169\mathbb{Z}} \\
    G &\cong \frac{\mathbb{Z}}{121\mathbb{Z}} \times \frac{\mathbb{Z}}{169\mathbb{Z}}
\end{align*}

\end{document}