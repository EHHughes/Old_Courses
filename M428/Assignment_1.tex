\documentclass{article}

\title{	
	\normalfont\normalsize 
	\rule{\linewidth}{0.5pt}\\ % Thin top horizontal rule
	\vspace{14pt} % Whitespace
	{\LARGE MATH428 Assignment 1 \\ % The assignment title
    \large \textit{} \\}
	\vspace{6pt} % Whitespace
	\rule{\linewidth}{1pt}\\ % Thick bottom horizontal rule
}

\author{Elliott Hughes}
\date{\normalsize\today}
\usepackage{tikz}
\usetikzlibrary{arrows,automata}
\usetikzlibrary{positioning}
\usetikzlibrary{arrows.meta,positioning}
\usepackage{mdframed}
\usepackage{amsmath}
\usepackage{amssymb}
\usepackage{graphicx}
\graphicspath{ {./Images/} }
\usepackage{commath}
\usepackage{textcomp}
\usepackage{gensymb}
\usepackage{float}
\usepackage{hyperref}
\usepackage[margin=1in]{geometry}
\usepackage{caption}
\usepackage{subcaption}
\usepackage{sectsty}
\usepackage{titlesec}

\begin{document}

\maketitle

\section*{Q1}
\subsection*{(a)}
($\rightarrow$) If for all $x \in X$, $\{x\}$ is the intersection of all its neighborhoods, then $\forall y \neq x$ 
it must be the case that $y \notin \cap_{i \in \mathcal{I}} N_i$, where $\{N_i:i \in \mathcal{I}\}$ is the 
set of all neighborhoods of $x$. Therefore 
there exists $N_x$ such that $y \notin N_x$. By relabeling one obtains that $X$ has the property 
$(*)$.

\paragraph{}
($\leftarrow$) Conversely, if $\forall y \neq x$ there exists $N_x$ a neighborhood such that $y \notin N_x$ 
then clearly the intersection of all neighborhoods of $x$ must only contain $x$ as desired. 
Thus (a) is equivalent to the property $(*)$.

\subsection*{(b)}
($\rightarrow$) If any finite set is closed then $\{x\}$ is closed for all $x \in X$. Consequently 
this set contains all of its limit points so for $y \neq x$ is not a limit point 
of this set. Therefore there exists some neighborhood of $y$ which has only trivial intersection 
with $\{x\}$ (e.g. $x$ is not in this neighborhood). By relabeling it follows that this space 
has the property $(*)$.

\paragraph{}
($\leftarrow$) For some finite set $F \subseteq X$ one can consider the complement $O = X\backslash F$. 
Given that $X$ has the property $(*)$ for 
any point $x \in O$ and any point $y \in F$ one can choose a neighborhood $N_{xy}$ such that 
$x \in N_{xy}$ and $y \notin N_{xy}$. Each of these neighborhoods contain an open set $O_{xy}$ 
such that $x \in O_{xy}$ 
and for a fixed $x$ one can take the intersection of the corresponding open sets for each $y \in F$. 
Since this a finite intersection of open sets this intersection is itself open. Since we chose a 
arbitrary $x$ it follows that for any $x \in O$ there exists an open set containing $x$ which has only trivial 
intersection with $F$ and so $O$ is open. Since $O$ is open it follows that $F$ is closed as required.

\subsection*{(c)}
Consider a topology on a set of two points $\{0,1\}$ given by $\tau = \{\emptyset,\{0,1\}\}$. For 
the only possible pair of distinct points (e.g. $0$ and $1$) neighborhoods of either of these 
points must contain an open set containing these points and thus must be supersets of 
$\{0,1\}$. It is thus clear that this topology cannot have the property $(*)$.

\paragraph{}
Consider the finite complement topology on $\mathbb{R}$. Clearly for any two distinct points 
$x,y \in \mathbb{R}$ one can define an open set (and thus a neighborhood) $N_x = \mathbb{R}\backslash \{y\}$ 
which contains $x$ and does not contain $y$. Consequently this topology has the property $(*)$ 
but it is also easy to show it is not Hausdorff. In particular, since any open set can only exclude a 
finite number of points all open sets are infinite and any two open sets have non-trivial intersection. 
Consequently for $x,y \in \mathbb{R}$ it is impossible to choose disjoint open sets $O_x$ and $O_y$ 
such that $x \in O_x$, $x \notin O_y$ and visa-versa.


\section*{Q2}

\subsection*{(a)}
Let $A$ be an infinite subset of $\mathbb{R}$. From Exercise 2.20 it follows that every point in 
$\mathbb{R}$ is a limit point of $A$ under $\tau_{co}$. Therefore $\bar{A} = \mathbb{R}$ 
in this topology so $A$ is dense in this topology.

\subsection*{(b)}
It will be most convenient to work with closed sets in this case, so consider $C$ a closed 
set induced by $(\mathbb{R},\tau_{co})$. Then for each element in $c \in C$ consider the roots 
of the polynomial $p(x) -c$. Clearly if $p(x) \neq c$ then this equation has a finite number (possibly zero) 
of roots and if $p(x) = c$ then the set of roots is the whole space $\mathbb{R}$. So for a 
finite set $C$, $p^{-1}(C)$ is either a finite set of points or $\mathbb{R}$ if $p(x) = c$ for 
$c \in C$. Since a finite set of points and $\mathbb{R}$ are closed in $(\mathbb{R},\tau_{co})$ 
it follows that $p^{-1}(C)$ is closed for all closed $C$ and so $p$ is continuous.

\paragraph{}
Consider $f(x) = |x|$. For $c$ in a closed set $C$ induced by $(\mathbb{R},\tau_{co})$, the 
equation $f(x) - c$ has at most two roots. Therefore $f^{-1}(C)$ is also a finite set and 
so $f$ is continuous in $(\mathbb{R},\tau_{co})$.

\subsection*{(c)}
Consider an open set in $(\mathbb{R},\tau_{co})$ given by $O_* = \mathbb{R}\backslash\{a_1,a_2,\dots,a_n\}$ for some $a_i \in \mathbb{R}$. For 
convenience we will assume without loss of generality that the complement set $\{a_1,a_2,\dots,a_n\}$ is ordered so that $a_i < a_{i+1}$ 
for each $1 \leq i \leq n$. Therefore one can also write this set as 

\begin{equation*}
    O_* = (-\infty,a_1) \cup (a_1,a_2) \cup \dots \cup (a_{n-1},a_n) \cup (a_n,\infty)
\end{equation*}

It is hopefully clear that $O_*$ is open in the Euclidean topology, so its preimage under $f$ will 
also be open (given $f$ is a continuous function from $(\mathbb{R},\tau_{E})$ to $(\mathbb{R},\tau_{E})$). 
Since the empty set and the whole space $\mathbb{R}$ are open sets in both topologies the 
pre-image of both sets under $f$ must also be open. 
Therefore for every open set in $O \in \tau_{co}$ the preimage of $O$ under $f$ is open 
in $(\mathbb{R},\tau_{E})$ and thus $f$ is a continuous function from $(\mathbb{R},\tau_{E})$ to $(\mathbb{R},\tau_{co})$.

\subsection*{(d)}
The step function

\begin{equation*}
    H(x) = 
    \begin{cases}
        0 & x < 0 \\
        1 & x \geq 0
    \end{cases}
\end{equation*}

The set $\{0\}$ is closed under the Euclidean topology, but its pre-image is $(-\infty,0)$. Since 
this set is not the complement of a set in $(\mathbb{R},\tau_{co})$ it is not closed. Therefore 
this function is not continuous.

\section*{Q3}
Without loss of generality let $T = \{0,1\}$.

\subsection*{(a)}
($\leftarrow$) Since $\{0\}$ and $\{1\}$ are open sets in $T$, it follows that for a continuous function $f:X\rightarrow T$ the 
sets $f^{-1}(\{0\})$ and $f^{-1}(\{1\})$ are open. These sets are trivially disjoint and their 
union is the whole set. Therefore if both are non-empty (e.g. the function is non-constant) then 
$X$ is disconnected. 

\paragraph{}
($\rightarrow$) The converse follows straightforwardly: if $X$ is disconnected and has an open partition formed 
from connected components $O_1$ and $O_2$ then one can define a function $f$ where $f(x) = 0$ 
if $x \in O_1$ and $f(x) = 1$ if $x \in O_2$. This function is continuous by construction and 
clearly non-constant. By contrapositives, if every continuous function $f$ from $X$ to $T$ is 
constant then $X$ is connected.

\paragraph{}
Since $X$ is disconnected if and only if there exists a non-constant continuous function from $X$ 
to $T$ it follows that $X$ is connected if and only if all continuous functions from $X$ to $T$ 
are constant.

\subsection*{(b)}
Consider a continuous function from $X\cup Y$ to $T$. Since $X$ and $Y$ 
are connected this function must be constant on both $X$ and $Y$. Without loss of generality 
let $f(X) = 1$. Then $f(X\cap Y) = 1$ and so $f(Y) = 1$. It follows then that $f(X \cup Y) =1$ 
and so $f$ is constant. Consequently $X \cup Y$ is connected.

\subsection*{(c)}
Let $f:B\rightarrow T$ be continuous. Under the induced subspace topology $A$ is a connected 
topological space and so the restriction of $f$ to $A$ must be constant. Then $f$ must be constant on $A$ and without loss of generality 
let $f(A) = 0$. Suppose that $f^{-1}(\{1\})$ is non-empty. Since $f$ is continuous it follows 
that this set is open. Since $B$ contains only limit points of $A$ (as $B \subseteq \bar{A}$) every 
open set of a point in $B\backslash A$ has non-trivial intersection with $A$. Therefore $f^{-1}(\{1\}) \cup A \neq \emptyset$ 
which is a contradiction. Thus $f^{-1}(\{1\})$ empty and so $f$ is constant on $B$. Since $f$ was 
arbitrary it follows from (a) that $B$ is connected.

\end{document}
