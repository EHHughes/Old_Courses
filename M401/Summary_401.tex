\documentclass{article}

\title{	
	\normalfont\normalsize 
	\rule{\linewidth}{0.5pt}\\ % Thin top horizontal rule
	\vspace{14pt} % Whitespace
	{\LARGE MATH401 Summary \\ % The assignment title
    \large \textit{} \\}
	\vspace{6pt} % Whitespace
	\rule{\linewidth}{1pt}\\ % Thick bottom horizontal rule
}

\author{Elliott Hughes}
\date{\normalsize\today}
\usepackage{tikz}
\usetikzlibrary{arrows,automata}
\usetikzlibrary{positioning}
\usetikzlibrary{arrows.meta,positioning}
\usepackage{mdframed}
\usepackage{amsmath}
\usepackage{mathtools}
\usepackage{amssymb}
\usepackage{graphicx}
\graphicspath{ {./Images/Assignment_2/} }
\usepackage{commath}
\usepackage{textcomp}
\usepackage{gensymb}
\usepackage{float}
\usepackage{hyperref}
\usepackage[margin=1in]{geometry}
\usepackage{caption}
\usepackage{subcaption}
\usepackage{sectsty}
\usepackage{titlesec}
\newtheorem{theorem}{Theorem}
\newtheorem{lemma}{Lemma}
\newtheorem{definition}{Definition}

\begin{document}

\maketitle

\section*{Flows and Interval Maps}
In this course we will study two main types of dynamical systems: flows and iterated maps. 
Throughout we will let $X$ be a compact metric space and $I_d$ the identity automorphism 
on this space.

\begin{definition}[Flows]
    A flow is a parameterized family of functions defined for all $X$ and for all $t \in mathbb{R}$ 
    then $\{\psi\}_{t\in\mathbb{R}}$ is a flow if and only if 

    \begin{itemize}
        \item $\psi_0 = I_d$
        \item $(\psi_t)^{-1} = \psi_{-t}$
        \item $\psi_{s+t} = \psi_s \circ \psi_t$
    \end{itemize}

    This flow also defines a unique differential equation \(\dot{x} = F(x) \coloneqq \frac{d}{dt}\psi_t(x)|_{t=0}\).
\end{definition}

\begin{definition}[Iterated Maps]
    A dynamical system induced by a iterated map is a parameterized family of functions defined for all $X$ and for all $n \in mathbb{N}$ 
    then $\{f^n\}_{n \in \mathbb{N}}$ is a family where 
    
    \begin{align*}
        f^0 &= I_d \\
        f^1 &= f \\
        f^2 & = f \circ f \\
        \dots& \\
        f^n(x) = f \circ f \circ \dots \circ f(x)
    \end{align*}

    If $f$ is invertible then the family can be extended to be defined over $\mathbb{Z}$ in an obvious manner.
\end{definition}

One can then define the orbit of a point $x_0$ under flows or iterated maps by considering the parameterized 
family of functions as a group or semi-group acting on $X$. If the orbit contains only one 
point, then $x_0$ is a fixed point. If the orbit is finite and of size $q$ then $x_0$ is a $q$-periodic point. 


\begin{definition}[Attracting Fixed Points of Maps]
    A fixed point $p^*$ of an iterated map $f$ is attracting if there exists some open neighborhood $N$ such that 

    \begin{equation*}
        x \in N qquad \lim_{n \rightarrow \infty}f^n(x) = p^*
    \end{equation*}

    $N$ is an attracting neighborhood and the largest connected $N$ is called the immediate basin of 
    attraction. 
\end{definition}

There is an obvious analogous definition for the fixed points of flows. We can now make further remarks 
about necessary conditions for fixed points to be attracting.

\begin{theorem}
    If $f$ is differentiable on $X$ and has a fixed point at $p^*$ then this fixed point is attracting 
    if $|f'(p^*)| < 1$.
\end{theorem}

A periodic orbit of size $q$ of a map is attracting if each $x_i \in \{x_1,x_2,\dots,x_q\}$ are 
attracting fixed points of $f^q$. Fortunately it is sufficient to instead verify the following 
simpler condition

\begin{definition}[Attracting Periodic Orbits of Maps]
    If $\Gamma = \{x_n\}_{n=1}^q$ is a periodic orbit of size $q$ and $f$ is differentiable at each point 
    in $\Gamma$ then $\Gamma$ is attracting if 

    \begin{equation*}
        \Pi_{x \in \Gamma}|f(x)'|<1
    \end{equation*}
\end{definition}

It is often desirable to know when periodic points of such maps are guaranteed to exist. Fortunately 
this is straightforward to demonstrate

\begin{theorem}
    Let $I$ be a subinterval of $X$ and $q > 0$. Then 

    \begin{enumerate}
        \item If $I \cap f^q(I) \neq \emptyset$ then $I$ contains a point fixed by $f^q$.
        \item If $I_0,I_1,\dots,I_k$ is a sequence of intervals such that each $I_{j+1} \subseteq I_j$ 
        for all $0 \leq j < q$ and $I_0 \subseteq f(I_{q-1})$ then $\bar{I_0}$ contains a point fixed 
        by $f^q$.
    \end{enumerate}
\end{theorem}

\section*{Hyperbolicity}
It is usually more convenient to work with the linearization of maps and/or flows rather than 
the full non-linear dynamics. Fortunately, this simplification is permissible near hyperbolic 
fixed points. 

\begin{definition}[Hyperbolic Fixed Points]
    A fixed point of a map is hyperbolic if the Jacobian derivative has no eigenvalues 
    of modulus 1. A fixed point of a flow is hyperbolic if the Jacobian derivative of the 
    vector field has eigenvalues with real part zero.
\end{definition}

\begin{theorem}[Hartman-Grobman for Maps]
    If $x^*$ is a hyperbolic fixed point of $f$ then there exists a neighborhood $N$ of $x^*$ 
    and a unique `near identity' homeomorphism $h:N \rightarrow \mathbb{R}^d$ such that $h(x^*) = 0$ 
    and $h(f(x)) = DT(x^*)h(x)$ for all $x \in N$.
\end{theorem}

An analogous result holds for flows. However, if we want to analyze the fixed points of flows, 
we will need more sophisticated machinery. In particular, it is useful to introduce the concept 
of variational equations.

\begin{lemma}[Variational Equations]
    Given a Linearization of a flow $\{\Psi_\tau(x_0)\}$ satisfy the coupled system of $d+d^2$ 
    equations. 

    \begin{align*}
        \dot{x} &=F(x) \quad x(0) = x_0 \\
        \dot{\Psi} = DF(x)\Psi \quad \Psi_0 = I_d
    \end{align*}
\end{lemma}

This matrix $\Psi$ is known as the Floquet matrix. In practice, we usually wish to use the Poincare 
map to find out information about the stability of periodic orbits. To begin, let us choose a 
Poincare section

\begin{definition}[Poincare Sections]
    Let $\Sigma$ be a $d-1$ dimensional surface (it is usually convenient to let this be the 
    zero set of some function $S$). This surface is a valid Poincare section if it is transverse to 
    the flow (that is $\nabla S \cdot F \neq 0$ for all $x$). Then the Poincare map is the first 
    return map on this section.
\end{definition}

From this we can obtain information about the Floquet matrix and visa-versa.

\begin{theorem}[Eigenvalue Equivalence Principle]
    If $\Gamma$ is a period $\tau$ orbit and $x_0$ is a point on valid Poincare section intersecting 
    this orbit then the set of eigenvalues of the linearized Poincare Map is equal to the eigenvalues of 
    the Floquet matrix less one eigenvalue equal to one.
\end{theorem}

\section*{Invariant Manifolds}

\begin{theorem}[Invariant Manifold Theorem]
Let $f$ be a $C^r$ diffeomorphism and let $x^*$ be a hyperbolic fixed point. Then there exists 
an open neighborhood $N$ of $x^*$ such that the local stable and unstable manifolds exist, have 
the same dimension as equivalent subspace and are tangent to that subspace.
\end{theorem}

\begin{definition}[Global Stable Manifold]
    The global stable manifold is given by 

    \begin{equation*}
        W^s(x^*) = \{x:f^n(x) \rightarrow x^* \quad \text{as} \quad n \rightarrow \infty\} = \cup_{n=0}f^{-n}W^s_{loc}(x^*)
    \end{equation*}
\end{definition}

if $W^s(x^*) \cup W^u(y^*)$ for two fixed points this implies a `heteroclinic connection' between 
these fixed points. If $y^* = x^*$ then this is a homoclinic orbit. For non-hyperbolic fixed 
points there is a similar result to the existence for theorems for manifolds we have used 
above.

\begin{theorem}[Center Manifold Theorem]
    Let $x^*$ be a non-hyperbolic fixed point of $C^r$ dynamical system. To the splitting $E^s \oplus E^u \oplus E^c$ 
    there correspond locally invariant manifolds $W^s$, $W^u$ and $W^c$ which are tangent to 
    their respective subspaces. Furthermore $W^s$ and $W^u$ are unique, but $W^c$ may not be unique.
\end{theorem}

\section*{Topological Conjugacy}

\begin{definition}[Topological Conjugacy]
    We say $(Y,g)$ and $(X,f)$ are topologically conjugate if there exists a homeomorphism $h:X\rightarrow Y$ 
    such that $h \circ f = g \circ h$. Since $h$ is invertible we can also write this as 
    $h \circ f \circ h^{-1} = g$.
\end{definition}

It is also useful to define the notion of an $\omega$ limit set.

\begin{definition}[$\omega$ limit set]
    The $\omega$ limit set of a point $x^*$ is 

    \begin{equation*}
        \omega(x^*) = \lim_{n\rightarrow \infty}\cap_{k=n}^\infty \bar{\mathcal{O}^+(f^k(x_0))}
    \end{equation*}
\end{definition}

This allows us to define a notion of an attracting invariant set

\begin{definition}[Attracting Invariant Sets]
    An invariant set $\Omega$ is attracting if there is an open set $U$ such that $\Omega \subset U$ 
    and for $x \in U \implies \omega(x) \subset \Omega$. The largest set such that this prior 
    property holds is called the basin of attraction.
\end{definition}

\section*{Chaos}

\begin{definition}[Chaos in the sense of Devaney]
    A dynamical system $(X,f)$ is called if it has an invariant set $\Omega$ such that 

    \begin{enumerate}
        \item Periodic points are dense in $\Omega$.
        \item There is at least one $x$ such that $\mathcal{O}^+(x)$ is dense in $\Omega$.
        \item The system restricted to $\Omega$, $f_\Omega$ has sensitive dependence to initial 
        conditions. There is a $\delta_0$ such that for any $x \in \Omega$ and $\epsilon > 0$ there 
        exists an $x' \in \Omega$ and $n \in \mathbb{N}$ such that $d(x,x') < \epsilon$ but $d(f^n(x),f^n(x')) > \delta_0$. 
    \end{enumerate}
\end{definition}

There is some redundancy in the chaos conditions. In particular we have that 

\begin{theorem}
    Let $X$ be complete, $f$ continuous and $\Omega$ a non-empty transitive invariant set. Then if 
    periodic points are dense in $\Omega$ then either $\Omega$ is a single periodic orbit or $f$ 
    is chaotic on $\Omega$.
\end{theorem}

There is a useful transitivity lemma 

\begin{lemma}[Transitivity Lemma]
    Let $\emptyset \neq \Omega \subseteq X$ be invariant, satisfy chaos conditions 1 and 2 and not 
    be a single periodic orbit. Then 

    \begin{enumerate}
        \item $\Omega$ has no isolated points.
        \item If $U$, $V$ with $U \cap \Omega \neq \emptyset$ and $V \cap \Omega \neq \emptyset$ 
        there is an $n \geq 0$ such that $\Omega \cap T^n(U) \cap V \neq \emptyset$. 
    \end{enumerate}
\end{lemma}

Finally we require the notion of a semi-conjugacy 

\begin{definition}
    A function $\psi:X \rightarrow Y$ is a semi-conjugacy if $\psi \circ f = g \circ \psi$ but 
    $\psi$ is not invertible.
\end{definition}

We can now construct a condition for chaos in one space implying chaos in the other

\begin{theorem}
    Let $\Omega$ be an chaotic invariant set for $(f,X)$ and let $\psi$ be a semi-conjugacy between 
    $X$ and $Y$. Then either $\psi(\Omega)$ is a single periodic orbit of $g$ in $Y$ or a chaotic 
    invariant set of $g$ in $Y$.
\end{theorem}

\end{document}
