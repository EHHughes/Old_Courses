\documentclass{article}

\usepackage{tikz}
\usetikzlibrary{arrows,automata}
\usetikzlibrary{positioning}
\usetikzlibrary{arrows.meta,positioning}
\usepackage{mdframed}
\usepackage{amsmath}
\usepackage{amssymb}
\usepackage{graphicx}
\graphicspath{ {./Images/ }}
\usepackage{commath}
\usepackage{textcomp}
\usepackage{gensymb}
\usepackage{float}
\usepackage{hyperref}
\usepackage[margin=1in]{geometry}
\usepackage{caption}
\usepackage{subcaption}
\usepackage{sectsty}
\usepackage{titlesec}

\begin{document}

\paragraph{}
To establish chaos we will wish to develop a semi-conjugacy from some simpler shift-space. Since 
we have already established the orbit of $x= 1/2$ divides the interval into a finite number of 
sub-intervals and that applying the map to these intervals induce a system of F-coverings. 
[INSERT INFO ABOUT THE COVERING HERE]

\paragraph{}
We can then construct a string space, using an alphabet where each element in the alphabet is the 
name of one of the intervals. One can exclude a set of two cylinders by excluding all cylinders of 
the form $I_jI_k$ where $f(I_j)$ does not cover $I_k$. This then leads to an adjacency matrix $A$ 
and a sub-shift of finite type $\Sigma_A$. If one could verify that $(\Sigma_A,\sigma)$ contains a 
chaotic invariant set $\Omega$ then this suggests we could hopefully construct a semi-conjugacy between these spaces. 

\paragraph{}
Assuming that $(\Sigma_A,\sigma)$ is chaotic then we construct a function 
$\psi: \Sigma_A \rightarrow X$. For some $s \in \Sigma_A$, we search for $x_0 \in X$ such that 
for each $x_n = f^n(x_0) \in \mathcal{O}(x)$, $x_n \in \bar{I_n}$ where $I_n$ is the interval in the 
n-th place of $s$. Lemma 20 guarantees that such a point exists and if it possible to show that 
this is unique then this produces a well-defined function. 

\paragraph{}
This function is obviously a semi-conjugacy if it is well-defined and continuous, so given these assumptions then 
theorem 17 guarantees that $\psi(\Omega)$ that this is either a chaotic invariant set or a 
single periodic orbit. If there are distinct periodic sequences that do not belong to the same orbit 
in $\Sigma_A$ then these $s_0,s_1$ must correspond to $x_0,x_1$ which are not part of the same orbit 
in $X$. If this is the case then $\psi(\Omega)$ must not be a single periodic orbit 
in $X$ and so $\psi(\Omega)$ must be chaotic. Thus we have established our desired result.

\end{document}
